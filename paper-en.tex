\documentclass[14pt]{mmcs-article}
\usepackage[russian]{babel}
\usepackage{amsmath, amsthm, amsfonts, amssymb}
\usepackage{listings, listings-rust}

\graphicspath{{paper_images/}}
\newtheorem{theorem}{Theorem}
\newtheorem{notice}{Notice}

% \newcounter{notice}[section]
% \newenvironment{notice}[1][]{\refstepcounter{notice}\par\medskip
%     \noindent \textbf{Замечание~\thenotice. #1} \rmfamily
% }
% {\medskip}

\newcounter{definition}[section]
\newenvironment{definition}[1][]{\refstepcounter{definition}\par\medskip
    \noindent \textbf{Definition~\thedefinition. #1} \rmfamily
}
{\medskip}

\begin{document}

\renewcommand{\mod}[1]{\textrm{mod}\ #1}


\section{Main definitions}

Let us give main definitions that are used onwards. We will use the definition of a bipartite graph based on a definition 7.1 given in a book % <<Дискретная Математика>> Я.М. Ерусалимского % Сослаться на книжку Татта (Tutte) 

\begin{definition}
    A bipartite graph is a triple $G(V, E, f)$
    ($V$ is a set of vertices,
    $E$ is a set of edges,
    $f$ is an incidence function mapping every edge to an ordered pair of vertices),
    such that:

    \begin{itemize}
        \item $V = A \cup B$ and $A \cap B = \emptyset$ ;
        \item $f: E \rightarrow A \times B$ ;
    \end{itemize}

\end{definition}

\begin{definition}
    Sequence of edges $\mu = (e_1, ..., e_d)$, such that $e_i \neq e_{i+1}$ for all $i < d$, is called a path with first vertex $v_0$ and last vertex $v_d$ on graph $G(V,E,f)$, if exists a sequence of vertices $(v_0, ..., v_d)$ such that $\forall i = 1,...,d:$ $(v_i, v_{i+1}) = f(e_i)$ or $(v_{i+1}, v_i) = f(e_i)$. 

\end{definition}

\begin{definition}
    A cycle is a path, for which the first and the last vertices are equal.
\end{definition}

\begin{definition}
    Let $G$ be a graph. The girth of $G$ is the length of it's shortest cycle.
\end{definition}

\begin{notice}
    Girth of any bipartite graph is even.
\end{notice}

It is known that graphs with higher girth have the most practical application. % Добавить цитату

\begin{definition}
    Let $\mu = (e_1, \dots, e_d)$ be a path on graph $G$. Characteristic of edges on path $\mu$ is the function $\chi_\mu: E \to \mathbb{Z}$ such that:

    \[
        \begin{array}{ll}
            \chi_{\mu}(e_1) = \left\{
                \begin{array}{ll}
                1,  & v_0 \in A; \\
                -1, & v_0 \in B. \\
                \end{array}
            \right. \\
            \chi_\mu(e_i) = -\chi_\mu(e_{i-1})\;\;\forall\, i \in 2, \dots, d\\
        \end{array}
    \]
\end{definition}

\section{Metagraphs}

Consider weighted bipartite graphs that can be used to build larger graphs with similar structure.

\begin{definition}
    Consider $G(V,E,f)$ a bipartite graph. A quadriple $G'(V,E,f,w)$ is called a metagraph, where $w: E \to \mathbb{Z}$ is a function mapping edges to their weights.
\end{definition}

Example \ref{image:2} shows metagraph  $G_1$ with three vertices and four edges.

\begin{figure}[H]
    \centering
    \begin{picture}(150,200)
        \put(75,165){\thicklines{\circle*{5}}}
        \put(70,170){$c_1$}
    
        \put(35,35){\thicklines{\circle*{5}}}
        \put(30,20){$i_1$}
    
        \put(115,35){\thicklines{\circle*{5}}}
        \put(110,20){$i_2$}
    
        \bezier{300}(75,165)(10,100)(35,35)
        \put(5,100){$+1$}

        \bezier{300}(75,165)(56,100)(35,35)
        \put(56,80){$0$}
    
        \bezier{300}(75,165)(140,100)(115,35)
        \put(125,100){$-1$}

        \bezier{300}(75,165)(94,100)(115,35)
        \put(87,80){$0$}
    \end{picture}
    \caption{ Metagraph with edges weights of +1, 0, 0, -1. }
    \label{image:2}
\end{figure}

\begin{definition}
    Let $r \in \mathbb{N}$, an $r$-expansion of metagraph $G(V,E,f,w)$ is a graph $G^{(r)}(V^{(r)}, E^{(r)}, f^{(r)})$, such that each vertex $v \in V$ corresponds to the set of vertices $ T^{(r)}(v) = \{ v^{(1)}, \dots, v^{(r)} \}$; each edge $e \in E$ corresponds to the set of edges $ T^{(r)}(e) =  \{ e^{(1)}, \dots, e^{(r)} \}$; and the incidence function is built according to the following rule: for each $e \in E$ such that $f(e) = (a, b)$, then $f^{(r)}(e^{(i)}) = (a^{(i)}, b^{(i + w(e) (\mod{r}))})$ $\forall i = 1, \dots, r$.
\end{definition}

\begin{notice}
    Sets of vertices and edges are constructed as follows:
    \[
        V^{(r)} = \bigcup_{v \in V} T^{(r)}(v);\;\;\;
        E^{(r)} = \bigcup_{e \in E} T^{(r)}(e).
    \]
\end{notice}

Example \ref{image:3} shows graph, built as a 4-expansion of metagraph $G_1$, shown in example \ref{image:2}.

\begin{figure}[H]
    \centering
    \begin{picture}(450,200)
        \put(75,165){\thicklines{\circle*{5}}}
        \put(70,170){$c_1$}
        \put(35,35){\thicklines{\circle*{5}}}
        \put(30,20){$i_1$}
        \put(115,35){\thicklines{\circle*{5}}}
        \put(110,20){$i_2$}

        \bezier{300}(75,165)(56,100)(35,35)
        \bezier{300}(75,165)(94,100)(115,35)
        \bezier{300}(75,165)(130,100)(135,35)
        \bezier{300}(175,165)(120,100)(115,35)

        \put(175,165){\thicklines{\circle*{5}}}
        \put(170,170){$c_1$}
        \put(135,35){\thicklines{\circle*{5}}}
        \put(130,20){$i_1$}
        \put(215,35){\thicklines{\circle*{5}}}
        \put(210,20){$i_2$}

        \bezier{300}(175,165)(156,100)(135,35)
        \bezier{300}(175,165)(194,100)(215,35)
        \bezier{300}(175,165)(230,100)(235,35)
        \bezier{300}(275,165)(220,100)(215,35)


        \put(275,165){\thicklines{\circle*{5}}}
        \put(270,170){$c_1$}
        \put(235,35){\thicklines{\circle*{5}}}
        \put(230,20){$i_1$}
        \put(315,35){\thicklines{\circle*{5}}}
        \put(310,20){$i_2$}

        \bezier{300}(275,165)(256,100)(235,35)
        \bezier{300}(275,165)(294,100)(315,35)
        \bezier{300}(275,165)(330,100)(335,35)
        \bezier{300}(375,165)(320,100)(315,35)


        \put(375,165){\thicklines{\circle*{5}}}
        \put(370,170){$c_1$}
        \put(335,35){\thicklines{\circle*{5}}}
        \put(330,20){$i_1$}
        \put(415,35){\thicklines{\circle*{5}}}
        \put(410,20){$i_2$}

        \bezier{300}(375,165)(356,100)(335,35)
        \bezier{300}(375,165)(394,100)(415,35)

        \bezier{700}(75,165)(245,100)(415,35)
        \bezier{700}(375,165)(205,100)(35,35)
    \end{picture}
    \caption{ 4-expansion of metagraph $G_1$. }
    \label{image:3}
\end{figure}

\begin{theorem}[on first vertices of paths]
    Let $\eta = (e_1, \dots, e_d)$ be a path with first vertex $v$ on metagraph $G$, then there are $r$ non-intersecting paths $\mu_1, \dots, \mu_r$ in $G^{(r)}$, with starting vertices $v^{(1)}, \dots, v^{(r)}$, such that for every path $\mu'=(e_1', \dots, e_d') \in\{\mu_1, \dots, \mu_r\}$,  \; $e_{j}' \in T^{(r)}(e_j)$ for every $j \in 1 \dots d$.
\end{theorem}

\begin{proof}
    The proof of this statement repeats the proof of theorem 1 in \cite{skorohodov_reachability_problem}.
\end{proof}

\begin{definition}
    Paths $\mu_i$ on graph $G^{(r)}$ are said to be corresponding to path $\eta$ on metagraph $G$ and vice versa.
\end{definition}

\begin{notice}
    If some path is a cycle on metagraph $G$, then it's corresponding paths on $G^{(r)}$ might not be cycles. Such a case is demonstrated by the following example.
\end{notice}

\textbf{Example 1.}

Consider graph on figure \ref{metagraph_3_expansion}. Path $(e_1, e_2)$ is a cycle on metagraph, but none of it's corresponding paths  $\mu_1 = (e^{(1)}_1, e^{(1)}_2)$, $\mu_2 = (e^{(2)}_1, e^{(2)}_2)$, $\mu_3 = (e^{(3)}_1, e^{(3)}_2)$  on 3-expansion is a cycle. Notice that concatenation of these three paths $\mu = (e^{(1)}_1, e^{(1)}_2, e^{(2)}_1, e^{(2)}_2, e^{(3)}_1, e^{(3)}_2)$ is a cycle with length 6. Cycle $\mu$ corresponds to cycle $(e_1, e_2, e_1, e_2, e_1, e_2)$ on metagraph.

\begin{figure}[H]
    \centering
    \begin{picture}(255,200)
        \put(0,165){$a.$}

        \put(35,165){\thicklines{\circle*{5}}}
        \put(30,170){$A$}
        \put(35,35){\thicklines{\circle*{5}}}
        \put(30,20){$B$}
    
        \bezier{300}(35,165)(10,100)(35,35)
        \put(0,100){$e_1$}
        \bezier{300}(35,165)(60,100)(35,35)
        \put(56,100){$e_ 2$}

        \put(110,165){$b.$}

        \put(145,165){\thicklines{\circle*{5}}}
        \put(140,170){$A^{(1)}$}
        \put(145,35){\thicklines{\circle*{5}}}
        \put(140,20){$B^{(1)}$}

        \thicklines
        \bezier{300}(145,165)(145,100)(145,35)
        \put(125,130){$e_0^{(1)}$}

        \bezier{300}(145,165)(165,100)(185,35)
        \put(147,85){$e_1^{(1)}$}
        \thinlines

        \put(185,165){\thicklines{\circle*{5}}}
        \put(180,170){$A^{(2)}$}
        \put(185,35){\thicklines{\circle*{5}}}
        \put(180,20){$B^{(2)}$}

        \bezier{300}(185,165)(185,100)(185,35)
        \put(165,130){$e_0^{(2)}$}
        
        \bezier{300}(185,165)(205,100)(225,35)
        \put(187,85){$e_1^{(2)}$}

        \put(225,165){\thicklines{\circle*{5}}}
        \put(220,170){$A^{(3)}$}
        \put(225,35){\thicklines{\circle*{5}}}
        \put(220,20){$B^{(3)}$}

        \bezier{300}(225,165)(225,100)(225,35)
        \put(205,130){$e_0^{(3)}$}

        \bezier{300}(225,165)(232,50)(145,35)
        \put(190,45){$e_1^{(3)}$}
    \end{picture}
    \caption{ a. Metagraph. b. 3-expansion of metagraph. }
    \label{metagraph_3_expansion}
\end{figure}

\begin{definition}
    Let $G(V, E, f, w)$ be a metagraph, $\eta = (e_1, ..., e_d)$ be a path on $G$.

    Characteristic of $\eta$ is
    \[
        \chi(\eta) = \sum_{i = 1}^d \chi_{\eta}(e_i) w(e_i).
    \]
\end{definition}

It is clear that the following theorem holds.

\begin{theorem} \label{glued_paths}
    Let $G$ be a metagraph, $\mu = (e_1, \dots, e_d)$ and $\eta = (u_1, \dots, u_k)$ be paths on $G$ and let $\xi = (e_1, \dots, e_d, u_1, \dots, u_k)$ also be a path on $G$; let $\mu' = (e_d, \dots, e_1)$ be an inversion path of $\mu$. Then the following relations are true:

    1. $ \chi(\xi) = \chi(\mu) + \chi(\eta)$,

    2. $ \chi(\mu') = -\chi(\mu)$.
\end{theorem}

\begin{theorem}[on last vertices of paths]\label{theorem_last_vertices}
    Let $\eta = (e_1, ..., e_d)$ be a path on metagraph $G$ with first vertex $x$ and last vertex $y$. Let $\mu' = (e'_1, ..., e'_d)$ be one of it's corresponding paths on $G^{(r)}$ with first vertex $x^{(i)}$.
    
    Then the last vertex of $\mu'$ is $y^{(i + \chi(\eta) (\mod{r}))}$.
    
\end{theorem}
    
\begin{proof}
    % Посмотреть, как по книжке Сосинского правильно описывать индукцию
    Conduct the proof by method of mathematical induction on the length of the path $|\eta| = d$.
    
    \begin{enumerate}
    \item $d = 1$, i.e. the path $\eta$ consist of only one edge $x$.
    
    If $x \in A$, then vertex $x^{(i)}$ is incident to the edge $e_1'$, for which $f^{(r)}(e_1') = (x^{(i)}, y^{(i + w(e_1) (\mod{r}))})$ hence the last vertex of the path $\mu_i$ is $y^{(i + w(e_1) (\mod{r}))} = y^{(i + \chi(\eta) (\mod{r}))}$.
    
    If $x \in B$, then vertex $x^{(i)}$ is incident to the edge $e_1'$, for which $f^{(r)}(e_1') = (y^{(i - w(e_1) (\mod{r}))}, x^{(i)})$ hence the last vertex of the path $\mu_i$ is $y^{(i - w(e_1) (\mod{r}))} = y^{(i + \chi(\eta) (\mod{r}))}$.
    
    \item Let for any $d \leq n$ the statement under the proof holds. We prove that it holds for $d = n + 1$.

    Suppose $\eta' = (e_1, ..., e_{d - 1})$ is a path. Let us call $x$ the last vertex of path $\eta$. According to the supposition of induction the last vertices of it's corresponding paths on expansion are $x^{(i + \chi(\eta')(\mod{r}))}$.
    
    If $x \in A$, then vertex $x^{(i + \chi(\eta')(\mod{r}))}$ is incident to the edge $f^{(r)}(e_d^{(i + \chi(\eta')(\mod{r}))}) = (x^{(i + \chi(\eta')(\mod{r}))}, y^{(i + \chi(\eta') + w(e_d) (\mod{r}))})$ hence the last vertex of the path $\mu_i$ is $y^{(i + \chi(\eta') + w(e_d) (\mod{r}))} = y^{(i + \chi(\mu) (\mod{r}))}$.
    
    If $x \in B$, then vertex $x^{(i + \chi(\eta')(\mod{r}))}$ is incident to the edge $f^{(r)}(e_d^{(i + \chi(\eta') - w(e_d) (\mod{r}))}) = (y^{(i + \chi(\eta') - w(e_d) (\mod{r}))}, x^{(i)})$ hence the last vertex of the path $\mu_i$ is $y^{(i + \chi(\eta') - w(e_d) (\mod{r}))} = y^{(i + \chi(\mu) (\mod{r}))}$.  
    
    \end{enumerate}  
\end{proof}

Theorems \ref{theorem_cycles} and \ref{theorem_wings} are direct corollaries of theorem \ref{theorem_last_vertices}.

\begin{theorem}[on cycles]\label{theorem_cycles}
    Let $\eta$ be a cycle on metagaph. Then it's corresponding paths on the expansion are cycles iff $\chi(\eta) = 0 (\mod{r})$.
\end{theorem}

\begin{theorem}[on wings of cycles]\label{theorem_wings}
    Let $\eta = (e_1, ..., e_d)$ and $\mu = (\epsilon_1, ..., \epsilon_{\delta})$ be different paths on metagraph $G$, with first vertex $a$, and last vertex $b$, $\chi(\eta) = \chi(\mu) (\mod{r})$, and let $\gamma = (e_1, \dots, e_d, \epsilon_{\delta}, \dots, \epsilon_1)$ be a path on $G$. Then paths on expansion which correspond to $\gamma$, are cycles.
\end{theorem}

\section{On graphs with given girth.}

% help us __to__ build?

Let $G(V, E, f, w)$ be a metagraph and $w(e_i) = x_i \; \forall i = 0, \dots, |E|$, where $x_i$ are variables that will help us build a graph with given girth.

% Поискать, как в англоязычной литературе называют не равенства с \neq

Since we have to build an expansion graph $G'$ with girth $g$, therefore for any cycle $\eta$ with length less than $g$ on metagraph $G$, assume inequality of the form $\chi(\eta) \neq 0$.

\begin{theorem}

Let $G$ be a metagraph. Then there exists an expansion of metagraph $G$ with girth equal to or greater than $g$ iff the following system of inequalities for a graph $G$ has a solution:

\begin{equation}
    \centering
    \left\{
        \begin{array}{ll}
            \chi(\eta) &\neq 0 (\mod{r}) \; \forall \, \eta \in \mathcal{H}_G(g).
        \end{array}
    \right.,
    \label{eqs:example}
\end{equation}
where $\mathcal{H}_G(g)$ is the set of all cycles on $G$ with length less then $g$.

\end{theorem}

\begin{proof}
    % exhausts?
    The construction of system (1) exhausts all cycles, hence the absence of a solution is equivalent to existence of a cycle with length less then $g$ on any expansion. Thus the existence of a solution for system (1) is equivalent to existence of an expansion with girth greater than $g$.
\end{proof}

% Добавить везде где надо (\mod{r})

For a given metagraph $G$, system \eqref{eqs:example} can be built using the following algorithm.

\textbf{Algorithm 1.}

Let $G(A \cup B, E, f, w)$ be a metagraph and let $w(e_i) = x_i \; \forall i = 0..|E|$.

% Нужно заменить g на какую-то не используемую выше букву и заменить во всей теореме

This algorithm is based on the construction of the set of labels.
Each such label is a quadruple $(v, l, ch, p)$,
$v$ is a vertex marked by this label;
where $l$ is a length of the path $\mu$ from starting vertex to the vertex $v$; 
$ch \in \mathbb{Z}$ characteristic of the path $\mu$;
$p \in V \cup \{ nil \}$ is an edge that was used to build this label, $nil \not\in V $ is a special value reserved for a starting label.

Assign the system of all inequalities $\mathcal{S}$ to be empty.
For each vertex $a \in A$:

\begin{itemize}
    \item Add starting label $(a, 0, 0, nil)$ to an empty set of all labels,
    \item Iterating by $l$ from $1$ to $g / 2$:
    \begin{itemize}
        \item For all labels $(v, l - 1, ch, p)$:
        \item
            For each edge $e \not= p$ incident to vertex $v$,
            build label $(v', l, ch', p)$, where $ch' = ch + (-1)^{l} w(p)$, $v'$ is a vertex different from $v$, incident to $p$.
            Add this label to the set of all labels.
    \end{itemize}
    \item for all pairs of labels $(v_1, l_1, ch_1, p_1)$ and $(v_2, l_2, ch_2, p_2)$ such that $v_1 = v_2$ and $l_1 = l_2$, add inequality $ch_1 \neq \ch_2$ to the system $\mathcal{S}$.
\end{itemize}

\begin{notice}
    Algorithm 1 builds cycles with two similar "wings", i.e. cycles that consist of two path of equal length. All cycles on bipartite graph can be represented in such a way, thus all cycles from $\mathcal{H}_G(g)$ on metagraph will be found by Algorithm 1.
\end{notice}

\textbf{Example.}

In figure \ref{neq_system_graph} the metagraph $G$ with for edges $e_1, e_2, e_3, e_4$ is shown. Weights of edges are variables $x_1, x_2, x_3, x_4$.

\begin{figure}[H]
    \centering
    \begin{picture}(150,200)
        \put(75,165){\thicklines{\circle*{5}}}
        \put(70,170){$v_1$}
    
        \put(35,35){\thicklines{\circle*{5}}}
        \put(30,20){$v_2$}
    
        \put(115,35){\thicklines{\circle*{5}}}
        \put(110,20){$v_3$}
    
        \bezier{300}(75,165)(10,100)(35,35)
        \put(5,100){$x_1$}

        \bezier{300}(75,165)(56,100)(35,35)
        \put(56,80){$x_2$}
    
        \bezier{300}(75,165)(140,100)(115,35)
        \put(125,100){$x_3$}

        \bezier{300}(75,165)(94,100)(115,35)
        \put(87,80){$x_4$}
    \end{picture}
    \caption{ Metagraph $G$, with variable weights of edges $x_1, x_2, x_3, x_4$. }
    \label{neq_system_graph}
\end{figure}

% "girth to be 6." ???
Let us apply Algorithm 1 for metagraph $G$ and girth to be 6. The set of labels, constructed by the Algorithm is shown in Table \ref{cycle_search_table_neq}. The system of inequalities (2) is constructed using these labels.

\begin{table}[H]
    \centering
    \begin{tabular}{ | c | c | c | }
        \hline
        $g = 0$            & $g = 1$               & $g = 2$                   \\ \hline
        $(0, v_1, 0, nil)$ & $(1, v_2,  x_1, e_1)$ & $(2, v_1,  x_1 - x_2, b)$ \\ \hline
                           & $(1, v_2,  x_2, e_2)$ & $(2, v_1,  x_2 - x_1, a)$ \\ \hline
                           & $(1, v_3,  x_3, e_3)$ & $(2, v_1,  x_3 - x_4, e)$ \\ \hline
                           & $(1, v_3,  x_4, e_4)$ & $(2, v_1,  x_4 - x_3, d)$ \\ \hline
    \end{tabular}
    \caption{ The set of labels constructed during the execution of the Algorithm 1. }
    \label{cycle_search_table_neq}
\end{table}

\begin{equation}
    \centering
    \left\{
        \begin{array}{ll}
            x_1 - x_2 &\neq 0             ; \\
            x_3 - x_4 &\neq 0             ; \\
            2 x_1 - 2 x_2 &\neq 0         ; \\
            2 x_3 - 2 x_4 &\neq 0         ; \\
            x_1 - x_2 - x_3 + x_4 &\neq 0 ; \\
            x_1 - x_2 + x_3 - x_4 &\neq 0 . \\
        \end{array}
        \label{eqs:cycle_search_neqs}
    \right.
\end{equation}

\section{Solving systems of inequalities}

% Абзац про минимизацию r 

% Still ???

Finding of the minimal solution for system \eqref{eqs:example}, is a computationally hard problem. It requires to enumerate all possible solutions and select the minimal one. Still, if system \eqref{eqs:example} has a solution then one of them can be found by the following algorithm.

%%% ^^^^^^^^^^

\textbf{Algorithm 2.}

\begin{enumerate}
    \item Let $e$ be the number of variables in a system of inequalities.
    \item If the system contains inequality $0 \neq 0$ then it has no solution, the algorithm halts.
    \item Let us pick all inequalities of the form $a x_e + b \neq 0$. Let $I$ be the set of all such inequalities.
    \item Let $v$ be a number such that $v \neq -b/a \forall a x_e + b \in I$, if $I = \varnothing$, then let $v = 0$. Such a number can be found because there is a finite amount of inequalities.
    \item Substitute variable $x_e$ with value $v$ in all inequalities and add identity $x_e = v$ to solution. By $v$ selection conditions, none of inequalities will turn into $0 \neq 0$ by such a substitution. 
    \item If $e = 1$ the algorithm halts, otherwise let $e = e - 1$ and return to step 3.
\end{enumerate}

\textbf{Пример.}

Let $x_1$ be $0$, substitute it in a system of inequalities.

\begin{equation}
    \centering
    \left\{
        \begin{array}{ll}
            - x_2 &\neq 0            ; \\
            x_3 - x_4 &\neq 0        ; \\
            - 2 x_2 &\neq 0          ; \\
            - 2 x_4 &\neq 0          ; \\
            - x_2 - x_3 + x_4 &\neq 0; \\
            - x_2 + x_3 - x_4 &\neq 0. \\
        \end{array}
    \right..
    \label{eqs:cycle_search_neqs_1}
\end{equation}

Variable $x_2$ can not be equal to $0$, because it violates inequality $- x_2 \neq 0$. Let $x_2$ be $1$, substitute it in a system of inequalities.

\begin{equation}
    \centering
    \left\{
        \begin{array}{ll}
            - 1 &\neq 0             ; \\
            x_3 - x_4 &\neq 0       ; \\
            - 2 &\neq 0             ; \\
            - 2 x_4 &\neq 0         ; \\
            - 1 - x_3 + x_4 &\neq 0 ; \\
            - 1 + x_3 - x_4 &\neq 0 . \\
        \end{array}
    \right..
    \label{eqs:cycle_search_neqs_2}
\end{equation}

Let $x_3$ be $0$, substitute it in a system of inequalities.

\begin{equation}
    \centering
    \left\{
        \begin{array}{ll}
            - 1 &\neq 0       ; \\
            - x_4 &\neq 0     ; \\
            - 2 &\neq 0       ; \\
            - 2 x_4 &\neq 0   ; \\
            - 1 + x_4 &\neq 0 ; \\
            - 1 - x_4 &\neq 0 . \\
        \end{array}
    \right..
    \label{eqs:cycle_search_neqs_3}
\end{equation}

Variable $x_4$ can not be equal to $0, 1, -1$, because it violates inequalities. Let $x_4$ be $2$, substitute it in a system of inequalities:

\begin{equation}
    \centering
    \left\{
        \begin{array}{ll}
            - 1 &\neq 0   ; \\
            - 2 &\neq 0   ; \\
            - 2 &\neq 0   ; \\
            - 4 &\neq 0   ; \\
              1 &\neq 0   ; \\
            - 3 &\neq 0   . \\
        \end{array}
    \right..
    \label{eqs:cycle_search_neqs_4}
\end{equation}

Given inequalities do not allow $r$ to be $1, 2, 3, 4$, thus the least possible value of $r$ is $5$. 5-expansion of metagraph with weights, obtained as a solution to system (1), is shown in figure \ref{neq_system_res}. Girth of this graph is equal to $6$, and one of the cyc;es with length $6$ is marked in bold.

\begin{figure}[H]
    \centering
    \begin{picture}(380,200)
        \put(50,165){\thicklines{\circle*{5}}}
        \put(35,35){\thicklines{\circle*{5}}}
        \put(65,35){\thicklines{\circle*{5}}}
    
        \bezier{300}(50,165)(42,100)(35,35)
        \bezier{300}(50,165)(57,100)(65,35)


        \put(120,165){\thicklines{\circle*{5}}}
        \put(105,35){\thicklines{\circle*{5}}}
        \put(135,35){\thicklines{\circle*{5}}}

        \bezier{300}(120,165)(127,100)(135,35)
        \bezier{300}(120,165)(197,100)(275,35)

        \put(190,165){\thicklines{\circle*{5}}}
        \put(175,35){\thicklines{\circle*{5}}}
        \put(205,35){\thicklines{\circle*{5}}}

        \bezier{300}(190,165)(217,100)(245,35)
        \bezier{300}(190,165)(267,100)(345,35)

        \put(260,165){\thicklines{\circle*{5}}}
        \put(245,35){\thicklines{\circle*{5}}}
        \put(275,35){\thicklines{\circle*{5}}}

        \bezier{300}(260,165)(252,100)(245,35)
        \bezier{300}(260,165)(267,100)(275,35)
        \bezier{300}(260,165)(287,100)(315,35)
        \bezier{600}(260,165)(172,50)(65,35)

        \put(330,165){\thicklines{\circle*{5}}}
        \put(315,35){\thicklines{\circle*{5}}}
        \put(345,35){\thicklines{\circle*{5}}}

        \bezier{300}(330,165)(322,100)(315,35)
        \bezier{300}(330,165)(337,100)(345,35)
        \bezier{600}(330,165)(182,150)(35,35)
        \bezier{600}(330,165)(242,50)(135,35)

        \thicklines
        \bezier{300}(50,165)(77,100)(105,35)
        \bezier{300}(120,165)(112,100)(105,35)
        \bezier{300}(120,165)(147,100)(175,35)
        \bezier{300}(190,165)(182,100)(175,35)
        \bezier{300}(190,165)(197,100)(205,35)
        \bezier{300}(50,165)(127,100)(205,35)
    \end{picture}
    \caption{ Expansion of a metagraph with weights obtained as a solution of a system of inequalities. }
    \label{neq_system_res}
\end{figure}

\begin{notice}
    The solution obtained as a result of Algorithm 2 might not give the minimal number of vertices after expansion. 
\end{notice}

\section{On selection of structure of metagraphs}

In previous section we showed a method of finding weights, that allow us build metagraph expansions with given girth.
But we have not suggested a way to select a structure of a metagraph such that it would be possible to find such weights, i.e. such that system (1) built by Algorithm 1 is non-contradicting.

\begin{definition}
    Let $G$ be a metagraph with variable weights. The girth of $G$ is the maximum among all girths of r-extensions of $G$.
\end{definition}

\begin{notice}
    Let us sequentially apply Algorithm 1 to $G$ with target girths of $4, 6, \dots, 2n$, where $2n$ is the first number for which algorithm couldn't build a solution. Then girth of metagraph $G$ is $2n - 2$.
\end{notice}

Let us consider metagraphs that are complete bipartite graphs.

\begin{definition}
    Let $G(V,E,f)$ be a bipartite graph, let $A$ and $B$ be it's parts. $G$ is complete iff for each pair of vertices $a \in A, b \in B$ exists an edge $e$, such that $f(e) = (a, b)$.
\end{definition}

Example of complete bipartite graph is shown in figure \ref{full_graph_3_by_9}. 

\begin{figure}[H]
    \centering
    \begin{picture}(300,200)
        \put(65,165){\thicklines{\circle*{5}}}
        \put(155,165){\thicklines{\circle*{5}}}
        \put(245,165){\thicklines{\circle*{5}}}

        \put(35,35){\thicklines{\circle*{5}}}
        \put(65,35){\thicklines{\circle*{5}}}
        \put(95,35){\thicklines{\circle*{5}}}

        \bezier{500}(35,35)(35,35)(65,165)
        \bezier{500}(65,35)(65,35)(65,165)
        \bezier{500}(95,35)(95,35)(65,165)
        \bezier{500}(35,35)(35,35)(155,165)
        \bezier{500}(65,35)(65,35)(155,165)
        \bezier{500}(95,35)(95,35)(155,165)
        \bezier{500}(35,35)(35,35)(245,165)
        \bezier{500}(65,35)(65,35)(245,165)
        \bezier{500}(95,35)(95,35)(245,165)

        \put(125,35){\thicklines{\circle*{5}}}
        \put(155,35){\thicklines{\circle*{5}}}
        \put(185,35){\thicklines{\circle*{5}}}

        \bezier{500}(125,35)(125,35)(65,165)
        \bezier{500}(155,35)(155,35)(65,165)
        \bezier{500}(185,35)(185,35)(65,165)
        \bezier{500}(125,35)(125,35)(155,165)
        \bezier{500}(155,35)(155,35)(155,165)
        \bezier{500}(185,35)(185,35)(155,165)
        \bezier{500}(125,35)(125,35)(245,165)
        \bezier{500}(155,35)(155,35)(245,165)
        \bezier{500}(185,35)(185,35)(245,165)

        \put(215,35){\thicklines{\circle*{5}}}
        \put(245,35){\thicklines{\circle*{5}}}
        \put(275,35){\thicklines{\circle*{5}}}

        \bezier{500}(215,35)(215,35)(65,165)
        \bezier{500}(245,35)(245,35)(65,165)
        \bezier{500}(275,35)(275,35)(65,165)
        \bezier{500}(215,35)(215,35)(155,165)
        \bezier{500}(245,35)(245,35)(155,165)
        \bezier{500}(275,35)(275,35)(155,165)
        \bezier{500}(215,35)(215,35)(245,165)
        \bezier{500}(245,35)(245,35)(245,165)
        \bezier{500}(275,35)(275,35)(245,165)
    \end{picture}
    \caption{ Complete bipartite graph with parts consisting of 3 and 9 vertices. }
    \label{full_graph_3_by_9}
\end{figure}

\begin{notice}
    Any complete metagraph that has at least 2 vertices in one part and 3 in another has girth less than or equal to 12. Let us consider metagraph $G$ in figure \ref{full_graph_2_by_3}. $\mu =(e_1, e_4, e_5, e_2, e_3, e_6, e_4, e_1, e_2, e_5, e_6, e_3)$ is a cycle. Its characteristic is euqal to $x_1 - x_4 + x_5 - x_2 + x_3 - x_6 + x_4 - x_1 + x_2 - x_5 + x_6 - e_3 = 0$. Thus paths that correspond to $\mu$ are cycles on r-expansions of $G$.

    Thereby to build r-extensions with girths greater than 12 non complete metagraphs must be used.
\end{notice}


\begin{figure}[H]
    \centering
    \begin{picture}(200,200)
        \put(65,165){\thicklines{\circle*{5}}}
        \put(155,165){\thicklines{\circle*{5}}}

        \put(35,35){\thicklines{\circle*{5}}}
        \put(110,35){\thicklines{\circle*{5}}}
        \put(185,35){\thicklines{\circle*{5}}}

        \bezier{500}(35,35)(35,35)(65,165)
        \put(40,140){$e_1$}
        \bezier{500}(110,35)(110,35)(65,165)
        \put(65,120){$e_2$}
        \bezier{500}(185,35)(185,35)(65,165)
        \put(85,150){$e_3$}

        \bezier{500}(35,35)(35,35)(155,165)
        \put(125,150){$e_4$}
        \bezier{500}(110,35)(110,35)(155,165)
        \put(145,120){$e_5$}
        \bezier{500}(185,35)(185,35)(155,165)
        \put(165,140){$e_6$}
    \end{picture}
    \caption{ Complete metagraph $G$ with parts consisting of 2 and 3 vertices. }
    \label{full_graph_2_by_3}
\end{figure}

\end{document}