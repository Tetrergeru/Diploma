\documentclass[a4paper,12pt]{article}
\usepackage[T1]{fontenc}
\usepackage[utf8]{inputenc}
\usepackage[english,russian]{babel}
\usepackage{amssymb,amsmath,amsthm,amsfonts}
\usepackage{graphics}
\usepackage{enumitem}
\usepackage[pdftex]{color}

\usepackage[left=2.5cm,right=1.5cm,top=2cm,bottom=2cm]{geometry} % поля страницы
\parindent1.25cm


\usepackage{makeidx}

\begin{document}

\Large
\baselineskip=8mm

{\bf УДК 519.1}

{\bf Регулярные периодические динамические ресурсные сети}

{\bf Скороходов~В.\,А., 
Свиридкин~Д.\,О.} 

(Южный федеральный университет, Ростов-на-Дону)


\begin{abstract}
В настоящей работе изучается модель распределения ресурсного потока в динамической периодической ресурсной сети. Основной задачей является разработка методов нахождения предельного состояния (распределения) ресурса в динамической ресурсной сети. Показано, что для регулярных периодических динамических сетей предельное состояние существует и является единственным, а для его нахождения можно использовать подходы, разработанные для динамических сетей.
\end{abstract}



\section{Введение}

\indent

Теория динамических потоков в сетях (теория динамических сетей), взяв своё начало ещё в работах Д.Р. Форда и Л.Р. Фалкерсона (см., например, \cite{SvSkor:FordFulkersonDyn}), продолжила развитие в работах Д.Б.~Орлина, Д.Е.~Аронсона,  М.В.~Фоноберовой, Д.Д.~Лозовану,  Б.~Клинца и др. (см. \cite{SvSkor:Aronson}-\cite{SvSkor:SkorChebot}). Динамическая сеть представляет сосбой ориентированную сеть, некоторые характеристики дуг которой (в частности, пропускные способности дуг) зависят от дискретного времени. Таким образом, поток в сети не является стационарным, а зависит от времени. В работах Я.М.~Ерусалимского, В.А.~Скороходова и М.В.~Кузьминовой (см. \cite{SvSkor:ErusSkor}-\cite{SvSkor:Kuzm}) описан общий подход к решению потоковых задач в динамических сетях, который состоит в построении вспомогательной <<статической>> сети, описывающей динамику изменения исходной сети, и сведении потоковой задачи в исходной динамической сети к аналогичной задаче на вспомогательной. 

Немного в стороне от классической теории потоков в сетях стоят ресурсные сети, введённые и довольно хорошо изученные О.П.~Кузнецовым и Л.Ю.~Жиляковой (см. \cite{SvSkor:Kuzn}-\cite{SvSkor:Zhil2}). Ресурсная сеть -- это сеть без источников, для каждой дуги которой указана пропускная способность, а для каждой вершины -- величина находящегося в ней ресурса. В каждый момент дискретного времени ресурс каждой вершины перераспределяется между смежными с ней вершинами по определённым правилам. Таким образом, между каждыми последовательными моментами времени по дугам сети проходит поток. При этом, правила функционирования сети таковы, что обязательно выполняются два условия. Первое -- это условие замкнутости сети, т.е. ресурс ни в какой вершине сети не добавляется извне и не исчезает. Второе -- условие неразрывности: ресурс, выходящий из вершины, вычитается из ее ресурса, а входящий в вершину, прибавляется к ее ресурсу.

В настоящей работе изучается модель распределения ресурсного потока в динамической периодической ресурсной сети. Основной задачей является разработка методов нахождения предельного состояния (распределения) ресурса в динамической ресурсной сети. Показано, что для регулярных динамических сетей предельное состояние существует и является единственным, а для его нахождения можно использовать подходы, разработанные для динамических сетей в \cite{SvSkor:ErusSkor}-\cite{SvSkor:Kuzm}.

\section{Основные понятия}

\indent

Приведем основные понятия, определения и утверждения \cite{SvSkor:ErusSkor}-\cite{SvSkor:Zhil2}.

{\bf Определение~1. }{\it 
Ресурсной сетью называют связную ориентированную сеть $G(X,E)$ (где $X=\{x_1,\dots,x_n\}$) без стоков, для каждой дуги $(x_i,x_j)$ которой указана пропускная способность $r_{ij}$, и задана вектор-функция ${Q}(t)=(q_1(t),\dots,q_n(t))$, где  $q_i(t)\geq 0$ для всех $i\in [1;n]_Z$.
}

Величина $q_i(t)$ называется количеством ресурса в вершине $x_i$ в момент времени $t$.

Для того, чтобы определить вектор-функцию ${Q}(t)$ задается вектор ${Q}(0)$ начального распределения ресурса в сети $G$ и указываются правила перераспределения ресурсов (правила функционирования сети):
\begin{equation}
\label{eq:mainOld}
q_i(t+1)=q_i(t)-\sum\limits_{j=1}^n F_{ij}(t)+\sum\limits_{j=1}^n F_{ji}(t)\;\; \forall\, i\in[1;n]_Z,
\end{equation}
где $F_{ij}(t)$ – величина ресурсного потока выходящего по дуге $(i,j)$ в момент времени $t$ определяется следующим образом:
$$F_{ij}(t)=\left\{
\begin{array}{ll}
r_{ij},& q_i>\sum\limits_{k=1}^n r_{ik};\\
\displaystyle \frac{r_{ij}}{\sum\limits_{k=1}^n r_{ik}}\cdot q_i(t),& q_i\leq\sum\limits_{k=1}^n r_{ik}
\end{array}
\right..$$

Величину суммарного ресурса сети обозначим через $W$, т.е. $W=\sum\limits_{i=1}^nq_i(0)$.

{\bf Определение~2. }{\it 
Состояние ${Q}(t)$ называется устойчивым, если выполняется ${Q}(t)={Q}(t+1)$.
}

Согласно правилам перераспределения ресурса если ${Q}(t)$ ус\-тойчиво, то для всех натуральных $i$ имеет место равенство ${Q}(t)={Q}(t+i)$.

{\bf Определение~3. }{\it  
Состояние ${Q}^*=(q^*_1,\dots,q^*_n)$ называется асимптотически достижимым из состояния ${Q}(0)$, если для каждого $i\in[1;n]_Z$ и всякого $\varepsilon>0$ существует $t_{\varepsilon}$ такое, что для всех $t>t_{\varepsilon}$ имеем место неравенство $|q^*-q_i(t)|<\varepsilon$.
}

{\bf Определение~4. }{\it 
Состояние ${Q}^*$ называется предельным, если оно либо устойчиво и существует такой момент времени $t$, что ${Q}^*={Q}(t)$, либо оно асимптотически достижимо из состояния ${Q}(0)$.
}



{\bf Определение~5. }{\it 
Ресурсную сеть будем называть эргодической, если она является сильно связной.
}

{\bf Определение~6. }{\it 
Эргодическую ресурсную сеть будем называть регулярной, если существует, по крайней мере, два цикла, длины которых являются взаимно простыми числами.
}

\section{Периодические динамические сети}

\indent

Пусть $G(X,E,r,D)$ -- регулярная периодическая динамическая ресурсная сеть, т.е. такая сеть, для каждой дуги $u$  которой в каждый момент времени $t$ указана величина $r_{ij}(t)$ -- пропускная способность дуги $(x_i,x_j)$ в момент $t$ и имеет место соотношение $r_{ij}(t)=r_{ij}(t+D)$. При этом полагаем, что время является дискретным и пропускные способности дуг сети $G$ не обращаются в ноль. Рассмотрим вопрос о существовании и единственности предельного состояния для таких сетей.

В работе \cite{SvSkor:SkorAbdur} изучена задача нахождения порогового значения в таких сетях. Для этих целей процесс перераспределения ресурсов в динамической ресурсной сети моделировался при помощи вспомогательной ресурсной сети большего размера, но на которой пропускные способности дуг не зависят от времени (см. также \cite{SvSkor:Skor1}, \cite{SvSkor:Kuzm}). Показано, что, несмотря на регулярность исходной сети, вспомогательная сеть обязательно является $D$-циклической, что существенно влияет на существование единственного предельного состояния в случае малого ресурса.  

Отметим, что поскольку для каждой дуги динамической ресурсной сети указывается периодическая зависимоть пропускной способности от времени, значит, фактически для такой сети задана периодическая последовательность матриц пропускных способностей ${R}(t)$. 

Для дальнейшего изложения введём в рассмотрение следующие обозначения:

	- ${c}(t)$ -- вектор суммарных пропускных способностей вершин в момент времени $t$, т.е. $c_i(t)=\sum\limits_{j=1}^n r_{ij}(t)$.
	
	- ${P}(t)$ -- стохастическая матрица, определяющая долевое распределение ресурса по дугам сети (см. \cite{SvSkor:KuznZhil1}) в момент времени $t$.


Для введённых величин основное правило функционирования ресурсной сети \eqref{eq:mainOld} для динамических ресурсных сетей можно записать в виде
    \begin{equation}
    \label{main_eq}
    	{Q}(t+1) = \max\{{Q}(t) - {c}(t), {0}\} + \min\{{Q}(t), {c}(t)\} \cdot {P}(t).
    \end{equation}
В соотношении \eqref{main_eq} минимум и максимум берутся поэлементно, т.е. если для некоторых векторов $\mathbf{x}$ и $\mathbf{y}$ вектор $\mathbf{z}=\max\{\mathbf{x},\mathbf{y}\}$, то это означает, что $z_i=\max\{x_i,y_i\}$ для всех $i\in[1;n]_Z$. Таким образом, каждая вершина $x_i$ $i\in [1;n]_Z$ в момент $t$ отдает не более  $c_i(t)$ ресурса, который распределяется пропорционально долям, определяемых матрицей ${P}(t)$.


Далее для удобства правило функционирования \eqref{main_eq} будем иногда записывать в сокращенной форме:
\begin{equation*}
	{Q}(t + 1) = \mathcal{A}_t({Q}(t)).
\end{equation*}

Отметим, что каждой паре $({c}, {P})$ можно однозначно поставить в соответствие ориентированную сеть $G(X,E_P)$, на которой пропускная способность каждой дуги $(x_i,x_j)$ определяется следующим правилом: $r_{ij} = P_{ij} \cdot c_i$.   

{\bf Лемма~1. }{\it 
	Пусть ${P}_1$ и ${P}_2$ --- регулярные стохастические матрицы одинакового порядка и вектор ${c}=(1,\dots,1)$, и пусть $G(X, E_1)$ и $G(X,E_2)$ -- графы, соответствующие парам $({c}, {P}_1)$ и $({c}, {P}_2)$ соответственно. Тогда для того, чтобы матрица ${P} \cdot {Q}$ была регулярной стохастической матрицей, достаточно, чтобы $E_1 \subseteq E_2$ или $E_2 \subseteq E_1$.
}

\textsc{Доказательство. }
	Не нарушая общности, полагаем $E_1 \subseteq E_2$ и $E_0=E_2\setminus E_1$. Также можно считать, что множества дуг $E_1$ и $E_2$ являются бинарными отношениями, определёнными на множестве вершин $X$.

   Рассмотрим граф $G(X,E_3)$, соответствующий произведению матриц $P_1\cdot P_2$. Множество дуг этого графа  получить как композицию отношений $E_1\circ E_2$ (см. \cite{SvSkor:kompozOtn}). Тогда $$E_1\circ E_2=E_1\circ (E_1\cup E_0)=E_1^2\cup (E_1\circ E_2).$$  
   
   Поскольку матрица $P_1^2$ является регулярной как степень регулярной матрицы (см. \cite{SvSkor:Gant1}), следовательно, соответствующий ей граф $G(X,E_1^2)$ содержит по крайней мере два цикла, длины которых взаимно простые (см. \cite{SvSkor:Skor2}). Отметим также, что граф $G(X,E_1^2)$ является частичным графом графа $G(X,E_3)$, а значит, и $G(X,E_3)$ содержит по крайней мере два цикла, длины которых взаимно простые. А поскольку $G(X,E_3)$ соответствует матрице $P_1\cdot P_2$, следовательно последняя является регулярной.
   
   Лемма доказана.


Следует отметить, что произведение регулярных матриц может и не быть регулярной матрицей. Покажем эту ситуацию на следующем примере.

{\bf Пример 1.}

Рассмотрим следующие регулярные матрицы
$$ A = \left(\begin{array}{cccc}
		0{,}5 & 0{,}5 & 0 & 0 \\
		0 & 0 & 0 & 1 \\
		1 & 0 & 0 & 0 \\
		0 & 0 & 1 & 0
	\end{array}\right),\;\;\;    
	B = \left(\begin{array}{cccc}
	0 & 1 & 0 & 0 \\
	0 & 0 & 1 & 0 \\
	0{,}5 & 0 & 0 & 0{,}5 \\
	0 & 0 & 1 & 0
	\end{array}\right).
$$

На рис.1 представлены графы $G_A$ и $G_B$, соответствующие матрицам $A$ и $B$.

\begin{center}
\begin{picture}(270,110)

\put(40,30){\circle*{3}}
\put(40,90){\circle*{3}}
\put(100,30){\circle*{3}}
\put(100,90){\circle*{3}}
\put(200,30){\circle*{3}}
\put(200,90){\circle*{3}}
\put(260,30){\circle*{3}}
\put(260,90){\circle*{3}}

\put(0,70){$G_A$}

\put(35,100){\oval(21.00,21.00)}
\put(45,95){\thicklines{\vector(-1,-1){5}}}
\put(40,90){\thicklines{\vector(1,0){60}}}
\put(100,90){\thicklines{\vector(-1,-1){60}}}
\put(40,30){\thicklines{\vector(1,0){60}}}
\put(100,30){\thicklines{\vector(-1,1){60}}}

\put(160,70){$G_B$}

\put(200,90){\thicklines{\vector(1,0){60}}}
\put(260,90){\thicklines{\vector(0,-1){60}}}
\put(260,30){\thicklines{\vector(-1,1){60}}}
\bezier{312}(260.00,30.00)(230.00,15.00)(200.00,30.00)
\put(255,27){\thicklines{\vector(2,1){5}}}
\bezier{312}(260.00,30.00)(230.00,45.00)(200.00,30.00)
\put(205,32){\thicklines{\vector(-2,-1){5}}}

\put(30,95){$1$}
\put(100,95){$2$}
\put(100,15){$3$}
\put(30,15){$4$}

\put(190,95){$1$}
\put(260,95){$2$}
\put(260,15){$3$}
\put(190,15){$4$}


\put(135,5){\makebox(0,0)[ct]{Рисунок~1 --- Графы $G_A$ и $G_B$.}}
\end{picture}
\end{center}

Произведения рассматриваемых матриц имеют вид
$$
 A \cdot B = \left(\begin{array}{cccc}
0 & 0{,}5 & 0{,}5 & 0 \\
0 & 0 & 1 & 0 \\
0 & 1 & 0 & 0 \\
0{,}5 & 0 & 0 & 0{,}5
\end{array}\right), \;\;\;   
B \cdot A = \left(\begin{array}{cccc}
0 & 0 & 0 & 1 \\
1 & 0 & 0 & 0 \\
0{,}25 & 0{,}25 & 0{,}5 & 0 \\
1 & 0 & 0 & 0
\end{array}\right).
$$

На рис.2 представлены графы $G_A$ и $G_B$, соответствующие матрицам $A$ и $B$.

\begin{center}
\begin{picture}(290,110)

\put(60,30){\circle*{3}}
\put(60,90){\circle*{3}}
\put(120,30){\circle*{3}}
\put(120,90){\circle*{3}}
\put(220,30){\circle*{3}}
\put(220,90){\circle*{3}}
\put(280,30){\circle*{3}}
\put(280,90){\circle*{3}}

\put(0,70){$G_{A\cdot B}$}

\put(60,90){\thicklines{\vector(1,0){60}}}
\put(60,90){\thicklines{\vector(1,-1){60}}}
\bezier{312}(120.00,30.00)(105.00,60.00)(120.00,90.00)
\put(115,80){\thicklines{\vector(1,2){4}}}
\bezier{312}(120.00,30.00)(135.00,60.00)(120.00,90.00)
\put(125,39){\thicklines{\vector(-1,-2){4}}}
\put(60,30){\thicklines{\vector(0,1){60}}}
\put(55,21){\oval(21.00,21.00)}
\put(64,27){\thicklines{\vector(-1,1){4}}}

\put(165,70){$G_{B\cdot A}$}

\bezier{312}(220.00,30.00)(205.00,60.00)(220.00,90.00)
\put(215,80){\thicklines{\vector(1,2){4}}}
\bezier{312}(220.00,30.00)(235.00,60.00)(220.00,90.00)
\put(225,39){\thicklines{\vector(-1,-2){4}}}
\put(280,90){\thicklines{\vector(-1,0){60}}}
\put(280,30){\thicklines{\vector(-1,1){60}}}
\put(280,30){\thicklines{\vector(0,1){60}}}
\put(284,20){\oval(21.00,21.00)}
\put(285,30){\thicklines{\vector(-1,0){4}}}


\put(50,95){$1$}
\put(120,95){$2$}
\put(120,15){$3$}
\put(50,15){$4$}

\put(210,95){$1$}
\put(280,95){$2$}
\put(280,15){$3$}
\put(210,15){$4$}


\put(145,5){\makebox(0,0)[ct]{Рисунок~2 --- Графы $G_{A\cdot B}$ и $G_{B\cdot A}$.}}
\end{picture}
\end{center}

Обе матрицы (и $A\cdot B$, и $B\cdot A$) не являются регулярными, поскольку 

1) соответствующие им графы не являются сильно связными, 

2) в отдельных компонентах сильной связности только по одному простому циклу длины 2. 

Для динамической сети рассмотрим её временную развёртку (см. \cite{SvSkor:Skor1}, \cite{SvSkor:SkorAbdur}) -- ориентированный граф $G'(X',E')$, такой что для каждой вершины $x\in X$ графа $G$ ставится в соответствие $D$ вершин $A_x=\{x^0,\dots, x^{D-1}\}$ на развёртке $G'$, для каждой дуги $u\in E$ (положим для определённости $u=(x,y)$) исходного графа $G$ ставится в соответствие $D$ дуг $\{u_0,\dots, u_{D-1}\}$ на развёртке $G'$ так, что $u_i=(x^i,y^{(i+1)\,\mathrm{mod}\,D})$ для всех $i\in[0;D-1]_Z$. Вес дуги $u_i\in E'$ полагается равным весу дуги $u\in E$ в момент времени $i$.

{\bf Замечание. }
{\it На развёртке $G'$ множество вершин разбивается на непересекающиеся множества $V_i = \{x^i_1,\dots,x^i_n\}$, которые мы будем называть $i$-м временным слоем.} 

Тогда при введении сплошной нумерации вершин матрица долевого распределения ресурса для развёртки $G'$ будет иметь следующий вид:
$$P_{G'}=\left(
\begin{array}{ccccc}
\Theta& P(0)& \Theta& \dots& \Theta \\
\Theta& \Theta& P(1)& \dots& \Theta \\
\vdots& \vdots& \vdots& \ddots& \vdots \\
\Theta& \Theta& \Theta& \dots& P(D-2)\\
P(D-1)& \Theta& \Theta& \dots& \Theta
\end{array}
\right),
$$ где $\Theta$ -- квадратная марица, состоящая из нулей. Отметим, что матрицу $P_{G'}$ можно также записать в виде $P_{G'}={P}\cdot {S}^n$, где ${P} = \mathrm{diag}\{{P}(0), {P}(1), \dots, {P}(D-1)\}$ -- блочно-диагональная матрица, а ${S}$ -- матрица оператора циклического сдвига вправо.

Таким образом, как и в работе \cite{SvSkor:SkorAbdur} процесс перераспределения ресурса между вершинами динамической сети будем моделировать аналогичным процессом на развёртке, при этом, функционирование вспомогательной ресурсной сети описывается схожим с \eqref{main_eq} соотношением
\begin{equation}
	\label{unroll_eq}
	{Q'}(t + 1) = \max\{{Q'}(t) - {C}, {0}\} \cdot {S}^n + \min\{{Q'}(t), {C}\} \cdot {P_{G'}}, 
	\end{equation}
здесь ${C}=({c}(0), {c}(1),\dots, {c}(D-1))$. Начальное состояние для развёртки задаётся следующим образом ${Q}'(0) = ({Q}(0), {Q}(1),\dots,$ ${Q}(D-1))$. 
	
	Единственным отличием от функционирования ресурсных сетей с постоянными пропускными способностями является наличие оператора $S^N$, отвечающего за перенос <<избыточного>> ресурса в вершинах на следующий временной слой в те же самые вершины.

	
\section{Малые ресурсы и предельное состояние}

\indent
	
Рассмотрим вопрос о существовании единственного предельного состояния в случае малого ресурса, т.е. когда $W\leq T$.

	Построим $D$ сетей с постоянными пропускными способностями следующим образом:
    
    Сеть $G_i$ ($i = 0,\dots, D-1)$ определяется парой $(c(i),B(i))$, где $B(i) = P(i)\cdot P(i+1)\cdot \ldots \cdot P(D-1) \cdot P(0) \cdot \ldots \cdot P(i-1)$.
    
    Каждая из этих сетей является регулярной, поскольку по лемме 1 каждая матрица $B(i)$, определяющая долевое распределение потока в сети $G_i$, является регулярной. В регулярных сетях при малых ресурсах существует единственное предельное распределение (см. \cite{SvSkor:Zhil1}).

{\bf Лемма~2. }{\it
    	Пусть $Q^*_i$ --- вектор предельного состояния при $W=1$ в сети $G_i$, тогда: 
    	\begin{equation}
    	  \label{eq:lemm_cycle}
    		Q^*_{i+1 (\mathrm{mod}\, D)} = Q^*_i \cdot P(i)
    	\end{equation}    	
}

\textsc{Доказательство. }
	
	Рассмотрим для сети $G_{i+1(\mathrm{mod}\, D)}$ в качестве начального состояния вектор $Q_{*i}(0)=Q^*_i\cdot P(i)$.  Тогда поскольку рассматривается случай малого ресурса, значит, $Q_{*i}(1)=\mathcal{A}\left(Q_{*i}(0)\right)=(Q^*_i\cdot P(i))\cdot B(i+1(\mathrm{mod}\, D))=$

пользуясь определением	$B(i)$, преобразуем к следующему виду

\noindent	$=(Q^*_i\cdot P(i))\cdot P(i+1(\mathrm{mod}\, D))\cdot \ldots \cdot P(D-1) \cdot P(0) \cdot \ldots \cdot P(i-1)\cdot P(i)=$

перегруппируем множители и получим

$=Q^*_i\cdot B(i)\cdot P(i)$.

Поскольку вектор $Q^*_i$ является предельным состоянием для сети $G_i$, тогда $Q^*_i\cdot B(i)=Q^*_i$. Таким образом получили
$$Q_{*i}(1)=Q^*_i\cdot P(i)=Q_{*i}(0).$$
Последнее означает, что состояние $Q_{*i}(0)=Q^*_i\cdot P(i)$ является устойчивым, а значит, предельным для сети $G_{i+1(\mathrm{mod}\, D)}$. Таким образом, в силу единственности предельного состояния имеет место равенство \eqref{eq:lemm_cycle}.
	
	Лемма доказана.


{\bf Теорема~1. }{\it 
  Вектор $Q'_{*1}=\left(Q^*_0,\dots,Q^*_{D-1}\right)$ является предельным состоянием для ресурсной сети $G'$ при $W=1$.
}

\textsc{Доказательство. }

Действительно, если рассмотреть вектор $Q'(0)=Q'_{*1}$ в качестве начального вектора для вспомогательной ресурсной сети $G'$, то следующее состояние $Q'(1)=\mathcal{A'}(Q'(0))=Q'_{*1}\cdot P_{G'}$, поскольку $W=1$. Однако, $$Q'_{*1}\cdot P_{G'}=\left(Q^*_{D-1}\cdot P(D-1),Q^*_{0}\cdot P(0),\dots,Q^*_{D-2}\cdot P(D-2)\right)=$$
по лемме 2 

\noindent $=\left(Q^*_0,\dots,Q^*_{D-1}\right)=Q'_{*1}=Q'(0)$, т.е. состояние $Q'_{*1}$ является устойчивым для сети $G'$, а значит, предельным.

Теорема доказана.

{\bf Замечание. }
{\it В данной работе мы рассматриваем сильно регулярные динамические периодические ресурсные сети, т.е. такие сети, для которых все матрицы $B(i)$ являются регулярными. Однако, для существования единственного предельного состояния на вспомогательной сети $G'$ при $W=1$ достаточно и более слабого условия: <<хотя бы одна из матриц $B(i)$ должна быть регулярной>>.}

{\bf Теорема~2. }{\it 
		В сети $G'$ при $W\leq T$ существует единственное предельное состояние вида $$Q^{\prime *}=\left(W\cdot Q^*_0, W\cdot Q^*_1, \dots, W \cdot Q^*_{D-1}\right).$$
}

\textsc{Доказательство. }

		Рассмотрим состояние сети $G'$ в момент $t$: $Q'(t) = Q'^* + \Delta(t)$. Отметим, что сумма всех компонент каждого вектора $\Delta(t)$ равна нулю, поскольку суммарная величина ресурса в сети не меняется с течением времени. 
		
		Рассмотрим процесс функционирования сети $G'$. Подставляя $Q'(t)$ в \eqref{unroll_eq}, имеем
	\begin{multline*}
		Q'^* + \Delta(t+1) = \max (Q'^* + \Delta(t) - C, 0)\cdot S^n +\\ + \min (Q'^* + \Delta(t), C)\cdot P_{G'}.
	\end{multline*}
		
		Группируя и вынося $Q'^*$ из минимума по правилу $\min(a + b, a + c) = a + \min(b, c)$, получим
		\begin{multline*}
		Q'^*+\Delta(t+1) = \max (\Delta(t) - (C - Q^*), 0)\cdot S^n +\\ +Q'^*+ \min (\Delta(t), C - Q^*)\cdot P_{G'}.
		\end{multline*}
		
		Учитывая, что $C \geq Q^*$, обозначим разность $C - Q^*$ через $C'$ и получим, что изменение значений $\Delta(t)$ описывается соотношением \eqref{delta_eq}, аналогичным \eqref{unroll_eq}, но с другими пропускными способностями вершин:
		\begin{equation}
			\label{delta_eq}
				\Delta(t+1) = \max (\Delta(t) - C', 0)\cdot S^n + \min (\Delta(t), C')\cdot P_{G'}.
		\end{equation}
		
		Докажем, что $\lim\limits_{t\to \infty}\Delta(t) = 0$. 
		
		Представим вектор $\Delta(t)$ в виде суммы $\Delta(t)=\Delta^+(t)+\Delta^-(t)$ -- по положительной и отрицательной частям $\Delta(t)$. Таким образом, $\Delta^+(t)\geq 0$ и $\Delta^-(t)\leq 0$. Тогда соотношение \eqref{delta_eq} примет следующий вид:
		\begin{multline*}
		\Delta(t+1) = \max(\Delta^+(t) - C', 0)\cdot S^n +\\ + \min (\Delta^+(t), C')\cdot P_{G'}  + \Delta^-(t) \cdot P_{G'}.
		\end{multline*} 

		Отметим, что сумма компонент вектора $\Delta^-(t)\cdot P_{G'}$ совпадает с суммой компонент $\Delta^-(t)$ и отрицательна. Сумму оставшихся положительных векторов обозначим через $\hat\Delta^+(n)$.
		
		Нам осталось показать, что через некоторое конечное время $\tau$ найдется такая такая вершина $x\in G'$, что $(\Delta^-(t+\tau)\cdot P_{G'})_x \neq 0$ и $(\hat\Delta^+(t+\tau))_x \neq 0$.		
		
 Предположим, что это не так, что для любого промежутка времени для каждой вершины $x$ либо $(\Delta^-(t+\tau)\cdot P_{G'})_x = 0$, либо $(\hat\Delta^+(t+\tau))_x = 0$. Отсюда следует, что имеет место следующее соотношение 
		\begin{equation}
		\label{delta_neg_eq}
		\Delta^-(t+1) = \Delta^-(t)\cdot P_{G'}.
		\end{equation}
		
		Таким образом, изменение значений вектор-функции $\Delta^-(t)$ описывается марковской цепью, которая определяется стохастической матрицей ресурсной сети $G'$. Такая ситуация соответствует условиям теоремы 1 для единичного ресурса в сети $G'$. Это означает, что существует предельное <<состояние>> $\Delta^-_* = \delta\cdot(Q^*_0,\dots$ $\dots, Q^*_{D-1})$, где $\delta=\sum\limits_{x\in G'}\Delta^-(0)$, при этом вектор $\Delta_*^-$ не имеет нулевых компонент. Следовательно, существует конечное число $\theta$ (более того, можно показать, что для регулярных сетей $\theta\leq n$) такое, что вектор $\Delta^-(t+\theta)=\Delta^-(t)\cdot (P_{G'})^{\theta}$ не содержит нулевых компонент. Получили противоречие.
			
		Таким образом, через каждые $n$ итераций найдется такая вершина $x\in G'$, что $(\Delta^-(t+n)\cdot P_{G'})_x \ne 0$ и $(\hat\Delta^+_x(t+n))_x \ne 0$. Следовательно, сумма отрицательных компонент вектор-функции $\Delta(t)$ уменьшается через каждые $n$ итераций. Сумма положительных компонент по абсолютному значению совпадает с суммой отрицательных. Таким образом, общая сумма модулей компонент век\-тор-функции $\Delta$ стремится к нулю. Последнее означает, что\\ $\lim\limits_{t\to \infty}\Delta(t) = 0$.  
		
		Теорема доказана.

Таким образом, поскольку процесс перераспределения ресурсов в сети $G'$ моделирует аналогичный процесс в исходной динамической сети $G$ (см. \cite{SvSkor:SkorAbdur}), значит, последовательность $Q^{\prime *}=\left(W\cdot Q^*_0, W\cdot Q^*_1, ... W \cdot Q^*_{D-1}\right)$ образует набор предельных состояний (см. \cite{SvSkor:Zhil1}) в исходной сети $G$.


	\section{Большие ресурсы и единственность предельного состояния}
	
	\indent
	
	Теперь рассмотрим вопрос о существовании единственного предельного состояния в случае большого ресурса, т.е. когда $W>T$.
		
{\bf Теорема~3. }{\it 
Для любой регулярной периодической динамической ресурсной сети $G$, для любого начального состояния при $W>T$ существует предельное состояние.
}		

\textsc{Доказательство. }

Рассмотрим предельное состояние $Q^{\prime *}$ для вспомогательной сети при $W=T$. Построим вспомогательную сеть $G''$ соответствующую паре $(C'',P'')$, где $C''=C-Q^{\prime *}$, а матрица $P''$ получена из марицы $P_{G'}$ образом: если $C''_i= 0$, то $P''_i=(P_{G'})_i$. В противном случае элементы i-той строки $P''$ определяется соотношением
$$P''_{ij}=\left\{\begin{array}{ll}
1,& j=    i+n\, (\mathrm{mod}\, D);\\
0,& j\neq i+n\, (\mathrm{mod}\, D).
\end{array}
\right.,$$ при этом величина $C''_i$ полагается равной $W-T$.

Сеть $G''$ получена из $G'$ удалением насыщенных ресурсным потоком дуг, а в случае, если после такого удаления из какой-то вершины не осталось выходящих дуг, то достраивается дуга, ведущая в вершину следующего слоя. Последняя дуга означает перенос <<оставшегося>> ресурса в вершине исходной динамической ресурсной сети. Таким образом, сеть $G''$ моделирует распределение <<осташегося>> ресурса величины $W-T$, т.е. величины превышения порогового значения.

Отметим, что сеть $G''$ разбивается на полуэргодические компоненты связности, состоящие из невозвратных вершин и изолированных (см. \cite{SvSkor:Skor2}) эргодических компонент, при этом, если изолированная компонента содержит хотя бы одну вершину множества $A_x$, то она содержит все вершины этого множества. Величина ресурса невозвратных вершин со временем станет раной нулю и весь оставшийся <<нераспределённым>> ресурс соберётся в изолированных эргодических компонентах. Таким образом, будем рассматривать ресурсный поток только на изолированных эргодических компонентах.

Для каждой такой изолированной компоненты если 

1. она является простым циклом, т.е. содержит вершины только одного множества $A_x$ и только достроенные дуги, обеспечивающие перенос ресурса на следующий временной слой, то предельное состояние в такой компоненте является единственным;

2. её суммарный ресурс равен нулю, то её предельное состояние является единственным (нулевым);

3. она не является простым циклом и её суммарный ресурс не равен нулю, то найдём для неё пороговое значение (см. \cite{SvSkor:SkorAbdur}, \cite{SvSkor:Skor}) и повторим действие, описанное для вспомогательной сети $G'$. Структура вспомогательной сети $G'$ такова, что повторяя такой процесс для оставшегося ресурса в итоге будут оставаться только простые циклы, проходящие по всем временным слоям.

Количество таких построений не превышает числа вершин исходной сети. Таким образом, предельное состояние ресурсной сети $G'$ может быть получено суммированием соответствующих компонент предельных состояний всех полученных в итоге вспомогательных сетей.

Теорема доказана.

		
Подход, применённый в доказательстве позволяет говорить о существовании предельного состояния в случае $W>T$, но не позволяет определять его. Предельное состояние в случае больших ресурсов, так же как и для не динамических ресурсных сетей, зависит от начального состояния. Для его определения необходимо знать, в каких пропорциях распределится <<остаток>> ресурса по изолированным эргодическим компонентам.
	
	
	\begin{thebibliography}{90}	\large
	
\bibitem{SvSkor:FordFulkersonDyn} Ford, L. R. Constructing maximal dynamic flows from static flows / L. R. Ford, D. R. Fulkerson // Operations Research. -- 1958. -- Vol. 6. -- P. 419-433.	
	
\bibitem{SvSkor:Aronson} Aronson, J.E. А survey of dynamic network flows / J. E. Aronson // Annals of Operations Research. -- 1989. -- No. 20. -- P. 1-66.

\bibitem{SvSkor:ErzTak} Erzin A.I., Takhonov I.I. The problem of finding of balanced flow // Journal of Applied and Industrial Mathematics. -- 2005, -- vol. VIII, -- No. 3(23), -- P. 58-68.

\bibitem{SvSkor:FonLozo1} Fonoberova, M. The maximum flow in dynamic networks / Fonoberova M., D. Lozovanu // Computer Science Journal of Moldova. -- 2004. -- No. 3 (36). -- P. 387-396.

\bibitem{SvSkor:FonLozo2} Fonoberova,~M. The minimum cost multicommodity flow problem in dy\-na\-mic networks and an algorithm for its solving / M.~Fonoberova, D.~Lo\-zo\-vanu // Computer Science Journal of Moldova. -- 2005. No. 1 (37). -- P. 29-36.

\bibitem{SvSkor:KlinzWoeg} Klinz, B. One, two, three, many, or: complexity aspects of dynamic network flows with dedicated arcs / B. Klinz, C. Woeginger // Operations Research Letters. -- 1998. -- No. 22. -- P. 119-127.

\bibitem{SvSkor:Orlin} Orlin, J. B. Maximum-throughput dynamic network flows / J.B. Orlin // Math. Progr. -- 1983. -- Vol. 27. -- P. 214-231.

\bibitem{SvSkor:SkorChebot} Skorokhodov~V.A. The Maximum Flow Problem in a Network with Special Conditions of Flow Distribution / V. A. Skorokhodov, A. S. Chebotareva // Journal of Applied and Industrial Mathematics. -- 2015. -- Vol. 9, No. 3. -- P. 435-446.	


\bibitem{SvSkor:ErusSkor} Ерусалимский,~Я.М. Графы с нестандартной достижимостью: задачи, приложения / Я.М.~Ерусалимский, В.А.~Скороходов, М.В.~Кузьминова, А.Г.~Петросян. -- Ростов-на-Дону: Южный федеральный университет, 2009. -- 195~с.

\bibitem{SvSkor:Skor1} Скороходов,~В.\,А. Потоки в сетях с меняющейся длительностью прохождения / В.\,А.~Скороходов // Известия ВУЗов. Северо-Кавказский регион. Естественные науки. -- 2011. -- №~1. -- С. 21-26.

\bibitem{SvSkor:Kuzm} Кузьминова,~М.В. Периодические динамические графы. Задача о максимальном потоке / М.В.~Кузьминова // Известия ВУЗов. Северо-Кавказский регион. Естественные науки. -- 2008. -- № 5. -- С. 16-20.

\bibitem{SvSkor:Kuzn} Kuznetsov,~O./,P. Nonsymmetric resource networks. The study of limit states / O./,P.~Kuznetsov // Management and Production Engineering Review. -- 2011. -- Vol. 2, No 3. -- P. 33-39.

\bibitem{SvSkor:KuznZhil1} Кузнецов,~О.П. Двусторонние ресурсные сети -- новая потоковая модель / О.П.~Кузнецов, Л.Ю.~Жилякова // Доклады АН. -- 2010. -- Т.~433, №5. -- С.~609-612.

\bibitem{SvSkor:Zhil0} Жилякова,~Л.Ю. Несимметричные ресурсные сети. I. Процессы стабилизации при малых ресурсах / Л.Ю.~Жилякова // Автоматика и телемеханика. -- 2011. -- №4. -- С.~133-143.

\bibitem{SvSkor:Zhil1} Жилякова,~Л.Ю. Эргодические циклические ресурсные сети. I. Колебания и равновесные состояния при малых ресурсах / Л.Ю.~Жилякова // Управление большими системами. -- 2013. -- № 43. -- С.~34-54.

\bibitem{SvSkor:Zhil2} Жилякова,~Л.Ю. Эргодические циклические ресурсные сети. II. Большие ресурсы / Л.Ю.~Жилякова // Управление большими системами. -- 2013.  -- № 45. -- С.~6-29.

\bibitem{SvSkor:SkorAbdur} Скороходов,~В.\,А. Динамические ресурсные сети. Случай малого ресурса / В.\,А.~Скороходов, Х.~Абдулрахман // Вестник ВГУ. Физика. Математика. -- 2018. -- № 4. -- С. 186-194.

\bibitem{SvSkor:kompozOtn}  Новиков,~Ф.\,А. Дискретная математика / Ф.\,А.~Новиков.  -- СПб.: Питер, 2013. -- 432~с.

\bibitem{SvSkor:Gant1} Гантмахер,~Ф.Р. Теория матриц / Ф.Р.~Гантмахер. -- М.: Физматлит, 2010. -- 560~с.

\bibitem{SvSkor:Skor2} Скороходов,~В.А. Устойчивость и стационарное распределение на графах с нестандартной достижимостью / В.А.~Скороходов // Известия ВУЗов. Северо-Кавказский регион. Естественные науки. -- 2007. -- №4. -- С.~17-21.

\bibitem{SvSkor:Skor} Скороходов,~В.\,А. Задача нахождения порогового значения в эргодической ресурсной сети / В.А.~Скороходов // Управление большими системами. Выпуск 63. -- М.: ИПУ РАН, 2016. -- С. 6-23.
		
	\end{thebibliography}

\newpage

{\bf Сведения об авторах}

1. Скороходов Владимир Александрович, доктор физ.-мат. наук., профессор Института математики, механики и компьютерных наук Южного федерального университета,\\ адрес: ул. Мильчакова, 8а, г. Ростов-на-Дону, 344090, Россия, \\e-mail: pdvaskor@yandex.ru

\vspace{4pt}
2. Свиридкин Дмитрий Олегович, студент Института математики, механики и компьютерных наук Южного федерального университета,\\ 
адрес: ул. Мильчакова, 8а, г. Ростов-на-Дону, 344090, Россия, \\e-mail: sv.l1@mail.ru

\end{document}
