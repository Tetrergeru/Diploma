\documentclass[14pt]{mmcs-article}
\usepackage[russian]{babel}
\usepackage{amsmath, amsthm, amsfonts, amssymb}
\usepackage{listings, listings-rust}

\graphicspath{{paper_images/}}
\newtheorem{theorem}{Theorem}
\newtheorem{notice}{Notice}

% \newcounter{notice}[section]
% \newenvironment{notice}[1][]{\refstepcounter{notice}\par\medskip
%     \noindent \textbf{Замечание~\thenotice. #1} \rmfamily
% }
% {\medskip}

\newcounter{definition}[section]
\newenvironment{definition}[1][]{\refstepcounter{definition}\par\medskip
    \noindent \textbf{Definition~\thedefinition. #1} \rmfamily
}
{\medskip}

\begin{document}

\renewcommand{\mod}[1]{\textrm{mod}\ #1}


\section{Main definitions}

Let us give main definitions that are used onwards. We will use the definition of a bipartite graph based on a definition 7.1 given in a book <<Дискретная Математика>> Я.М. Ерусалимского % Сослаться на книжку Татта (Tutte) 

\begin{definition}
    A bipartite graph is a triple $G(V, E, f)$
    ($V$ is a set of vertices,
    $E$ is a set of edges,
    $f$ is an incidence function mapping every edge to an ordered pair of vertices),
    such that:

    \begin{itemize}
        \item $V = A \cup B$ and $A \cap B = \emptyset$ ;
        \item $f: E \rightarrow A \times B$ ;
    \end{itemize}

\end{definition}

\begin{definition}
    Sequence of edges $\mu = (e_1, ..., e_d)$, such that $e_i \neq e_{i+1}$ for all $i < d$, is called a path with first vertex $v_0$ and last vertex $v_d$ on graph $G(V,E,f)$, if exists a sequence of vertices $(v_0, ..., v_d)$ such that $\forall i = 1,...,d:$ $(v_i, v_{i+1}) = f(e_i)$ or $(v_{i+1}, v_i) = f(e_i)$. 

\end{definition}

\begin{definition}
    A cycle is a path, for which the first and the last vertices are equal.
\end{definition}

\begin{definition}
    Let $G$ be a graph. The girth of $G$ is the length of it's shortest cycle.
\end{definition}

\begin{notice}
    Girth of any bipartite graph is even.
\end{notice}

It is known that graphs with higher girth have the most practical application. % Добавить цитату

\begin{definition}
    Let $\mu = (e_1, \dots, e_d)$ be a path on graph $G$. Characteristic of edges on path $\mu$ is the function $\chi_\mu: E \to \mathbb{Z}$ such that:

    \[
        \begin{array}{ll}
            \chi_{\mu}(e_1) = \left\{
                \begin{array}{ll}
                1,  & v_0 \in A; \\
                -1, & v_0 \in B. \\
                \end{array}
            \right. \\
            \chi_\mu(e_i) = -\chi_\mu(e_{i-1})\;\;\forall\, i \in 2, \dots, d\\
        \end{array}
    \]
\end{definition}

\section{Metagraphs}

Consider weighted bipartite graphs that can be used to build larger graphs with similar structure.

\begin{definition}
    Consider $G(V,E,f)$ a bipartite graph. A quadriple $G'(V,E,f,w)$ is called a metagraph, where $w: E \to \mathbb{Z}$ is a function mapping edges to their weights.
\end{definition}

 \ref{image:2} представлен метаграф  $G_1$ с тремя вершинами и четырьмя дугами.

\begin{figure}[H]
    \centering
    \begin{picture}(150,200)
        \put(75,165){\thicklines{\circle*{5}}}
        \put(70,170){$c_1$}
    
        \put(35,35){\thicklines{\circle*{5}}}
        \put(30,20){$i_1$}
    
        \put(115,35){\thicklines{\circle*{5}}}
        \put(110,20){$i_2$}
    
        \bezier{300}(75,165)(10,100)(35,35)
        \put(5,100){$+1$}

        \bezier{300}(75,165)(56,100)(35,35)
        \put(56,80){$0$}
    
        \bezier{300}(75,165)(140,100)(115,35)
        \put(125,100){$-1$}

        \bezier{300}(75,165)(94,100)(115,35)
        \put(87,80){$0$}
    \end{picture}
    \caption{ Метаграф с весами дуг +1, 0, 0, -1.. }
    \label{image:2}
\end{figure}

\begin{definition}
    Пусть $r \in \mathbb{N}$, тогда $r$-расширением метаграфа $G'(V,E,f,w)$ назовём граф $G^{(r)}(V^{(r)}, E^{(r)}, f^{(r)})$ построенный следующим образом: для каждой вершины $v \in V$ ставится в соответствие множество вершин $\{ v^{(1)}, ..., v^{(r)} \}$; для каждой дуги $e \in E$ ставится в соответствие множество дуг $\{ e^{(1)}, ..., e^{(r)} \}$, при этом если $f(e) = (a, b)$, тогда $f^{(r)}(e^{(i)}) = (a^{(i)}, b^{(i + w(e) (\mod{r}))})$ $\forall i = 1,...,r$.
\end{definition}

\begin{notice}
    Множества вершин и дуг расширения устроены следующим образом:
    \[
        V^{(r)} = \bigcup_{v \in V} \{ v^{(1)}, ..., v^{(r)} \};\;\;\;
        E^{(r)} = \bigcup_{e \in E} \{ e^{(1)}, ..., e^{(r)} \}.
    \]
\end{notice}

На рисунке. \ref{image:3} изображён граф, полученный 4-расширением метаграфа $G_1$, изображённого на рисунке \ref{image:2}.

\begin{figure}[H]
    \centering
    \begin{picture}(450,200)
        \put(75,165){\thicklines{\circle*{5}}}
        \put(70,170){$c_1$}
        \put(35,35){\thicklines{\circle*{5}}}
        \put(30,20){$i_1$}
        \put(115,35){\thicklines{\circle*{5}}}
        \put(110,20){$i_2$}

        \bezier{300}(75,165)(56,100)(35,35)
        \bezier{300}(75,165)(94,100)(115,35)
        \bezier{300}(75,165)(130,100)(135,35)
        \bezier{300}(175,165)(120,100)(115,35)

        \put(175,165){\thicklines{\circle*{5}}}
        \put(170,170){$c_1$}
        \put(135,35){\thicklines{\circle*{5}}}
        \put(130,20){$i_1$}
        \put(215,35){\thicklines{\circle*{5}}}
        \put(210,20){$i_2$}

        \bezier{300}(175,165)(156,100)(135,35)
        \bezier{300}(175,165)(194,100)(215,35)
        \bezier{300}(175,165)(230,100)(235,35)
        \bezier{300}(275,165)(220,100)(215,35)


        \put(275,165){\thicklines{\circle*{5}}}
        \put(270,170){$c_1$}
        \put(235,35){\thicklines{\circle*{5}}}
        \put(230,20){$i_1$}
        \put(315,35){\thicklines{\circle*{5}}}
        \put(310,20){$i_2$}

        \bezier{300}(275,165)(256,100)(235,35)
        \bezier{300}(275,165)(294,100)(315,35)
        \bezier{300}(275,165)(330,100)(335,35)
        \bezier{300}(375,165)(320,100)(315,35)


        \put(375,165){\thicklines{\circle*{5}}}
        \put(370,170){$c_1$}
        \put(335,35){\thicklines{\circle*{5}}}
        \put(330,20){$i_1$}
        \put(415,35){\thicklines{\circle*{5}}}
        \put(410,20){$i_2$}

        \bezier{300}(375,165)(356,100)(335,35)
        \bezier{300}(375,165)(394,100)(415,35)

        \bezier{700}(75,165)(245,100)(415,35)
        \bezier{700}(375,165)(205,100)(35,35)
    \end{picture}
    \caption{ Метаграф с весами дуг +1, 0, 0, -1.. }
    \label{image:3}
\end{figure}

\begin{theorem}[О начальных вершинах путей]
    Пусть $\eta = (e_1, \dots, e_d)$ ~-- путь с начальной вершиной $v$ на метаграфе $G$, тогда на графе $G^{(r)}$ существуют $r$ попарно не пересекающихся путей $\mu_1 \dots \mu_r$ с начальными вершинами $v^{(1)}, ..., v^{(r)}$, таких что для каждого пути $\mu'=(e_1',\dots,e_d') \in\{\mu_1,\dots,\mu_r\}$ справедливо, что для всех $j\in[1;d]_N$ $e_{j}' \in T^{(r)}(e_j)$.
\end{theorem}

\begin{proof}
    Доказательство этого утверждения практически дословно повторяет доказательство теоремы 1 в \cite{skorohodov_reachability_problem}.
\end{proof}

\begin{definition}
    Будем говорить, что пути $\eta$ на метаграфе $G$ соответствуют пути $\mu_i$ на графе $G^{(r)}$ и наоборот.
\end{definition}

\begin{notice}
    Из того, что некоторый путь на метаграфе $G$ является циклом не следует, что соответствующие ему пути на $G^{(r)}$ являются циклами. Рассмотрим эту ситуацию на следующем примере.
\end{notice}

\end{document}