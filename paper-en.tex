\documentclass[14pt]{mmcs-article}
\usepackage[russian]{babel}
\usepackage{amsmath, amsthm, amsfonts, amssymb}
\usepackage{listings, listings-rust}

\graphicspath{{paper_images/}}
\newtheorem{theorem}{Theorem}
\newtheorem{notice}{Notice}

% \newcounter{notice}[section]
% \newenvironment{notice}[1][]{\refstepcounter{notice}\par\medskip
%     \noindent \textbf{Замечание~\thenotice. #1} \rmfamily
% }
% {\medskip}

\newcounter{definition}[section]
\newenvironment{definition}[1][]{\refstepcounter{definition}\par\medskip
    \noindent \textbf{Definition~\thedefinition. #1} \rmfamily
}
{\medskip}

\begin{document}

\renewcommand{\mod}[1]{\textrm{mod}\ #1}


\section{Main definitions}

Let us give main definitions that are used onwards. We will use the definition of a bipartite graph based on a definition 7.1 given in a book <<Дискретная Математика>> Я.М. Ерусалимского % Сослаться на книжку Татта (Tutte) 

\begin{definition}
    A bipartite graph is a triple $G(V, E, f)$
    ($V$ is a set of vertices,
    $E$ is a set of edges,
    $f$ is an incidence function mapping every edge to an ordered pair of vertices),
    such that:

    \begin{itemize}
        \item $V = A \cup B$ and $A \cap B = \emptyset$ ;
        \item $f: E \rightarrow A \times B$ ;
    \end{itemize}

\end{definition}

\begin{definition}
    Sequence of edges $\mu = (e_1, ..., e_d)$, such that $e_i \neq e_{i+1}$ for all $i < d$, is called a path with first vertex $v_0$ and last vertex $v_d$ on graph $G(V,E,f)$, if exists a sequence of vertices $(v_0, ..., v_d)$ such that $\forall i = 1,...,d:$ $(v_i, v_{i+1}) = f(e_i)$ or $(v_{i+1}, v_i) = f(e_i)$. 

\end{definition}

\begin{definition}
    A cycle is a path, for which the first and the last vertices are equal.
\end{definition}

\begin{definition}
    Let $G$ be a graph. The girth of $G$ is the length of it's shortest cycle.
\end{definition}

\begin{notice}
    Girth of any bipartite graph is even.
\end{notice}

It is known that graphs with higher girth have the most practical application. % Добавить цитату

\begin{definition}
    Let $\mu = (e_1, \dots, e_d)$ be a path on graph $G$. Characteristic of edges on path $\mu$ is the function $\chi_\mu: E \to \mathbb{Z}$ such that:

    \[
        \begin{array}{ll}
            \chi_{\mu}(e_1) = \left\{
                \begin{array}{ll}
                1,  & v_0 \in A; \\
                -1, & v_0 \in B. \\
                \end{array}
            \right. \\
            \chi_\mu(e_i) = -\chi_\mu(e_{i-1})\;\;\forall\, i \in 2, \dots, d\\
        \end{array}
    \]
\end{definition}

\section{Metagraphs}

Consider weighted bipartite graphs that can be used to build larger graphs with similar structure.

\begin{definition}
    Consider $G(V,E,f)$ a bipartite graph. A quadriple $G'(V,E,f,w)$ is called a metagraph, where $w: E \to \mathbb{Z}$ is a function mapping edges to their weights.
\end{definition}

%%%

Example\ref{image:2} shows metagraph  $G_1$ with three vertices and four edges.

\begin{figure}[H]
    \centering
    \begin{picture}(150,200)
        \put(75,165){\thicklines{\circle*{5}}}
        \put(70,170){$c_1$}
    
        \put(35,35){\thicklines{\circle*{5}}}
        \put(30,20){$i_1$}
    
        \put(115,35){\thicklines{\circle*{5}}}
        \put(110,20){$i_2$}
    
        \bezier{300}(75,165)(10,100)(35,35)
        \put(5,100){$+1$}

        \bezier{300}(75,165)(56,100)(35,35)
        \put(56,80){$0$}
    
        \bezier{300}(75,165)(140,100)(115,35)
        \put(125,100){$-1$}

        \bezier{300}(75,165)(94,100)(115,35)
        \put(87,80){$0$}
    \end{picture}
    \caption{ Metagraph with edge weights of +1, 0, 0, -1. }
    \label{image:2}
\end{figure}

\begin{definition}
    Let $r \in \mathbb{N}$, a $r$-expansion of metagraph $G'(V,E,f,w)$ is a graph $G^{(r)}(V^{(r)}, E^{(r)}, f^{(r)})$ where each vertex $v \in V$ corresponds to the set of vertices $\{ v^{(1)}, ..., v^{(r)} \}$; each edge $e \in E$ corresponds to the set of edges $\{ e^{(1)}, ..., e^{(r)} \}$, and given $f(e) = (a, b)$, then $f^{(r)}(e^{(i)}) = (a^{(i)}, b^{(i + w(e) (\mod{r}))})$ $\forall i = 1,...,r$.
\end{definition}

\begin{notice}
    Sets of vertices and edges are constructed as follows:
    \[
        V^{(r)} = \bigcup_{v \in V} \{ v^{(1)}, ..., v^{(r)} \};\;\;\;
        E^{(r)} = \bigcup_{e \in E} \{ e^{(1)}, ..., e^{(r)} \}.
    \]
\end{notice}

Example \ref{image:3} shows graph, built as a 4-expansion of metagraph $G_1$, shown in example \ref{image:2}.

\begin{figure}[H]
    \centering
    \begin{picture}(450,200)
        \put(75,165){\thicklines{\circle*{5}}}
        \put(70,170){$c_1$}
        \put(35,35){\thicklines{\circle*{5}}}
        \put(30,20){$i_1$}
        \put(115,35){\thicklines{\circle*{5}}}
        \put(110,20){$i_2$}

        \bezier{300}(75,165)(56,100)(35,35)
        \bezier{300}(75,165)(94,100)(115,35)
        \bezier{300}(75,165)(130,100)(135,35)
        \bezier{300}(175,165)(120,100)(115,35)

        \put(175,165){\thicklines{\circle*{5}}}
        \put(170,170){$c_1$}
        \put(135,35){\thicklines{\circle*{5}}}
        \put(130,20){$i_1$}
        \put(215,35){\thicklines{\circle*{5}}}
        \put(210,20){$i_2$}

        \bezier{300}(175,165)(156,100)(135,35)
        \bezier{300}(175,165)(194,100)(215,35)
        \bezier{300}(175,165)(230,100)(235,35)
        \bezier{300}(275,165)(220,100)(215,35)


        \put(275,165){\thicklines{\circle*{5}}}
        \put(270,170){$c_1$}
        \put(235,35){\thicklines{\circle*{5}}}
        \put(230,20){$i_1$}
        \put(315,35){\thicklines{\circle*{5}}}
        \put(310,20){$i_2$}

        \bezier{300}(275,165)(256,100)(235,35)
        \bezier{300}(275,165)(294,100)(315,35)
        \bezier{300}(275,165)(330,100)(335,35)
        \bezier{300}(375,165)(320,100)(315,35)


        \put(375,165){\thicklines{\circle*{5}}}
        \put(370,170){$c_1$}
        \put(335,35){\thicklines{\circle*{5}}}
        \put(330,20){$i_1$}
        \put(415,35){\thicklines{\circle*{5}}}
        \put(410,20){$i_2$}

        \bezier{300}(375,165)(356,100)(335,35)
        \bezier{300}(375,165)(394,100)(415,35)

        \bezier{700}(75,165)(245,100)(415,35)
        \bezier{700}(375,165)(205,100)(35,35)
    \end{picture}
    \caption{ 4-expansion of metagraph. }
    \label{image:3}
\end{figure}

\begin{theorem}[About the first vertices of paths]
    Let $\eta = (e_1, \dots, e_d)$ be a path with first vertex $v$ on metagraph $G$, then there are $r$ non-intersecting paths $\mu_1 \dots \mu_r$ in $G^{(r)}$, with starting vertices $v^{(1)}, ..., v^{(r)}$, such that for every path $\mu'=(e_1',\dots,e_d') \in\{\mu_1, \dots, \mu_r\}$,  \; $e_{j}' \in T^{(r)}(e_j)$ for every $j \in 1 \dots d$.
\end{theorem}

\begin{proof}
    The proof of this statement repeats the proof of theorem 1 in \cite{skorohodov_reachability_problem}.
\end{proof}

\begin{definition}
    Let us call paths $\mu_i$ on graph $G^{(r)}$ corresponding to path $\eta$ on metagraph $G$ and vice versa.
\end{definition}

\begin{notice}
    If some path is a cycle on metagraph $G$ it's corresponding paths on $G^{(r)}$ might not be cycles. Such case is demonstrated in the following example. 
\end{notice}

\textbf{Example 1.}

Consider graph on image \ref{metagraph_3_expansion}. Path $(e_1, e_2)$ is a cycle on metagraph, but none of it's corresponding paths  $\mu_1 = (e^{(1)}_1, e^{(1)}_2)$, $\mu_2 = (e^{(2)}_1, e^{(2)}_2)$, $\mu_3 = (e^{(3)}_1, e^{(3)}_2)$  on 3-expansion is a cycle. Notice that concatenation of these three paths $\mu = (e^{(1)}_1, e^{(1)}_2, e^{(2)}_1, e^{(2)}_2, e^{(3)}_1, e^{(3)}_2)$ is a cycle with length 6. Cycle $\mu$ corresponds to cycle $(e_1, e_2, e_1, e_2, e_1, e_2)$ on metagraph.

\begin{figure}[H]
    \centering
    \begin{picture}(255,200)
        \put(0,165){$a.$}

        \put(35,165){\thicklines{\circle*{5}}}
        \put(30,170){$A$}
        \put(35,35){\thicklines{\circle*{5}}}
        \put(30,20){$B$}
    
        \bezier{300}(35,165)(10,100)(35,35)
        \put(0,100){$e_1$}
        \bezier{300}(35,165)(60,100)(35,35)
        \put(56,100){$e_ 2$}

        \put(110,165){$b.$}

        \put(145,165){\thicklines{\circle*{5}}}
        \put(140,170){$A^{(1)}$}
        \put(145,35){\thicklines{\circle*{5}}}
        \put(140,20){$B^{(1)}$}

        \thicklines
        \bezier{300}(145,165)(145,100)(145,35)
        \put(125,130){$e_0^{(1)}$}

        \bezier{300}(145,165)(165,100)(185,35)
        \put(147,85){$e_1^{(1)}$}
        \thinlines

        \put(185,165){\thicklines{\circle*{5}}}
        \put(180,170){$A^{(2)}$}
        \put(185,35){\thicklines{\circle*{5}}}
        \put(180,20){$B^{(2)}$}

        \bezier{300}(185,165)(185,100)(185,35)
        \put(165,130){$e_0^{(2)}$}
        
        \bezier{300}(185,165)(205,100)(225,35)
        \put(187,85){$e_1^{(2)}$}

        \put(225,165){\thicklines{\circle*{5}}}
        \put(220,170){$A^{(3)}$}
        \put(225,35){\thicklines{\circle*{5}}}
        \put(220,20){$B^{(3)}$}

        \bezier{300}(225,165)(225,100)(225,35)
        \put(205,130){$e_0^{(3)}$}

        \bezier{300}(225,165)(232,50)(145,35)
        \put(190,45){$e_1^{(3)}$}
    \end{picture}
    \caption{ a. Metagraph. b. 3-expansion of metagraph. }
    \label{metagraph_3_expansion}
\end{figure}

\begin{definition}
    Let $G(V, E, f, w)$ be a metagraph, $\eta = (e_1, ..., e_d)$ is a path on $G$.

    Characteristic of $\eta$ is
    \[
        \chi(\eta) = \sum_{i = 1}^d \chi_{\eta}(e_i) w(e_i).
    \]
\end{definition}

Clearly, the following statement is true

\begin{theorem} \label{glued_paths}
    Let $G$ be a metagraph, $\mu = (e_1, \dots, e_d)$ and $\eta = (u_1, \dots, u_k)$ be paths on $G$ and let $\xi = (e_1, \dots, e_d, u_1, \dots, u_k)$ also be a path on $G$; let $\mu' = (e_d, \dots, e_1)$ be an inversion of $\mu$. Thus the following relations are true:

    1. $ \chi(\xi) = \chi(\mu) + \chi(\eta)$,

    2. $ \chi(\mu') = -\chi(\mu)$.
\end{theorem}

\begin{theorem}[About the last vertices of paths]
    Let $\eta = (e_1, ..., e_d)$ be a path on metagraph $G$, it's first vertex is $x$, it's last vertex is $y$. Let $\mu' = (e'_1, ..., e'_d)$ be on of it's corresponding paths on $G^{(r)}$, it's first vertex is $x^{(i)}$.
    
    Then the last vertex of $\mu'$ is $y^{(i + \chi(\eta) (\mod{r}))}$.
    
\end{theorem}
    
\begin{proof}
    Доказательство проведём по индукции по длине пути $|\eta| = d$.
    
    Пусть $d = 1$, тогда путь состоит их одной дуги $e_1$.
    
    Если $x \in A$, тогда вершина $x^{(i)}$ инцидентна дуге $e_1'$, для которой $f^{(r)}(e_1') = (x^{(i)}, y^{(i + w(e_1) (\mod{r}))})$ то есть, конечной вершиной пути $\mu_i$ является $y^{(i + w(e_1) (\mod{r}))} = y^{(i + \chi(\eta) (\mod{r}))}$.
    
    Если $x \in B$, тогда вершина $x^{(i)}$ инцидентна дуге $e_1'$, для которой $f^{(r)}(e_1') = (y^{(i - w(e_1) (\mod{r}))}, x^{(i)})$ то есть, конечной вершиной пути $\mu_i$ является $y^{(i - w(e_1) (\mod{r}))} = y^{(i + \chi(\eta) (\mod{r}))}$.
    
    Пусть условия теоремы выполняются для всякого $d < n$, тогда рассмотрим случай $d = n$.
    
    Рассмотрим путь $\eta' = (e_1, ..., e_{d - 1})$. Обозначим его последнюю вершину $x$. Тогда по предположению индукции последние вершины соответствующих ему путей на метаграфе ~-- это $x^{(i + \chi(\eta')(\mod{r}))}$.
    
    Если $x \in A$, тогда вершина $x^{(i + \chi(\eta')(\mod{r}))}$ инцидентна дуге \\ $f^{(r)}(e_d^{(i + \chi(\eta')(\mod{r}))}) = (x^{(i + \chi(\eta')(\mod{r}))}, y^{(i + \chi(\eta') + w(e_d) (\mod{r}))})$ и конечная \\ вершина пути $\mu_i$ ~-- $y^{(i + \chi(\eta') + w(e_d) (\mod{r}))} = y^{(i + \chi(\mu) (\mod{r}))}$.
    
    Если $x \in B$, тогда вершина $x^{(i + \chi(\eta')(\mod{r}))}$ инцидентна дуге \\ $f^{(r)}(e_d^{(i + \chi(\eta') - w(e_d) (\mod{r}))}) = (y^{(i + \chi(\eta') - w(e_d) (\mod{r}))}, x^{(i)})$ и конечная вершина пути $\mu_i$ ~-- $y^{(i + \chi(\eta') - w(e_d) (\mod{r}))} = y^{(i + \chi(\mu) (\mod{r}))}$.    
\end{proof}

\end{document}