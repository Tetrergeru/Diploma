\documentclass[14pt]{mmcs-article}
\usepackage[russian]{babel}
\usepackage{amsmath, amsthm, amsfonts, amssymb}

% После кванторов ставить отступ
%% Починить \mod (\mathsc)

\graphicspath{{images/}}

\begin{document}

\section*{Введение}

\textbf{Определение 1.}

\textsl{Двудольным графом} будем называть пару $\langle V,E \rangle$, такую, что:

\begin{itemize}
    \item $V = A \cup B$ и $A \cap B = \emptyset$ ;
    \item $e = (a, b), a \in A, b \in B \forall e \in E$;
\end{itemize}

Где $V$ ~--- множество вершин, разбитое на два непересекающихся подмножества $A$ и $B$.
$E$ ~--- множество дуг. Дуга ~--- это упорядоченная пара вершин, называемых инцидентными этой дуге.

\textbf{Определение 2.}

% пример

Пусть $G$ ~--- граф $G = \langle A \cup B, E \rangle$.
Упорядоченную последовательность дуг $\mu = (e_1, ..., e_d)$ будем называть путём, если у дуги $e_i$, где $i = 2, ..., d-1$, одна инцидентная вершина общая с дугой $e_{i-1}$, а другая ~--- общая с дугой $e_{i+1}$.

\textbf{Определение 3.}

Вершину дуги $e_1$, не являющаяся общей с дугой $e_2$, называют первой вершиной пути.

Вершину дуги $e_d$, не являющаяся общей с дугой $e_{d-1}$, называют последней вершиной пути.

\textbf{Определение 5.}

С каждым путёю $\mu$ свяжем характеристическую функцию $\chi(\mu)$.
Так, если $v_0$ ~--- первая вершина пути, то
\[
    \chi_{\mu}(e_1) =
    \left\{
        \begin{array}{ll}
        1,  & v_0 \in A;\\
        -1, & v_0 \in B. \\
        \end{array}
    \right..
\]

$\forall i \in 2, ..., d$, если $e_1 = (a, b)$, то
\[
    \chi_{\mu}(e_i) =
    \left\{
        \begin{array}{ll}
        1,  & a \text{ инцидентна дуге } e_{i-1};\\
        -1, & \text{в противном случае}. \\
        \end{array}
    \right..
\]

\textbf{Определение 4.}

Циклом будем называть путь, у которого совпадают первая и последняя вершины.

\textbf{Определение 6.}

\textsl{Обхватом графа} называют длину его минимального цикла.

\textbf{Замечание 1.}

Отметим, что обхват любого двудольного графа является чётным числом, большим, чем два.

\textbf{Замечание 2.}

Известно, что на практике для кодирования эффективнее использовать графы с большим обхватом.

\pagebreak
\section*{Метаграфы}

\textbf{Определение 6.}

\textsl{Метаграфом} будем называть тройку $\langle A \cup B,E,w \rangle$, такую, что:

\begin{itemize}
    \item $\langle A \cup B,E \rangle$ ~--- двудольный граф;
    \item $w: E \rightarrow \mathbb{Z}$ ~--- отображение задающее веса дуг.
\end{itemize}

На (рис. \ref{image:2}) представлен пример метаграфа с тремя вершинами и четырьмя дугами.

%% ---------- Кратные дуги

\textbf{Замечание.}

Использованное определение метаграфа не позволяет рассмотреть случаи кратных дуг. Однако это не составляет проблемы, так как кратные дуги порождают циклы малой длины.

\begin{figure}[H]
    \centering
    \includegraphics[scale=0.4]{Fig_2.png}
    \caption{ Метаграф с весами дуг +1, 0, 0, -1.. }
    \label{image:2}
\end{figure}

Пусть $G$ ~--- метаграф, $G = \langle V, E, w \rangle$, и задано число $r \in \mathbb{N}$. Построим двудольный граф $G^{(r)} = \langle V', E' \rangle$ по следующим правилам:

\begin{itemize}
    \item Каждой вершине $v \in V$ поставим в соответствие множество вершин
    \[
        T^{(r)}_v = \{ v^{(j)} \}_{j = 1}^r
    \]
    Будем говорить, что вершины из этого множества соответствуют вершине $v$ и наоборот;

    \item Каждой дуге $e \in E$ (для определённости будем считать,что $e = (a, b)$) ставится в соответстие множество дуг
    \[
        R^{(r)}_e = \{ (a^{(j)}, b^{(j + w(e) (\mod{r}))}) | j = 1..r \}
    \]
    Будем говорить, что дуги из множества $R^{(r)}_e$ соответствуют дуге $e$ и наоборот;
\end{itemize}
Положим
\[
    V' = \bigcup_{v \in V} T^{(r)}_v; E' = \bigcup_{e \in E} R^{(r)}_e.
\]

\textbf{Определение 7.}

Граф $G^{(r)}$, построенный в ходе работы Алгоритма 1. будем называть расширением метаграфа $G$ в $r$ раз.

На (рис. \ref{image:3}) изображён граф, полученный расширением метаграфа из (рис. \ref{image:2}) в 4 раза.

\begin{figure}[H]
    \centering
    \includegraphics[scale=0.4]{Fig_1.png}
    \caption{ Метаграф с весами дуг +1, 0, 0, -1.. }
    \label{image:3}
\end{figure}

\textbf{Теорема 1.}

Пусть $\eta = (e_1, ..., e_d)$ ~--- путь на метаграфе $G$, тогда на графе $G^{(r)}$ есть $r$ попарно не пересекающихся путей вида $\mu_j = (e'_1, ..., e'_d)$, таких что $e'_i \in R^{(r)}_{e_i}$.

\textbf{Доказательство.}

% Попросить ссылку на статью

Следует из определения расширения метаграфа.

\qed

\textbf{Определение 8.}

Будем говорить, что путям $\mu$ на графе $G^{(r)}$ соответствует путь $\eta$ на метаграфе $G$ и наоборот.

\textbf{Замечание 2.}

Из того, что некоторый путь на метаграфе $G$ является циклом не следует, что соответствующие ему пути на $G^{(r)}$ являются циклами.

$/*$ Надо добавить пример, сейчас он есть только на бумажечке $*/$

\textbf{Определение 7.}

Пусть $G = \langle V, E, w \rangle$ ~--- метаграф.

Характеристикой пути $\eta = (e_1, ..., e_d)$ будем называть

\[
    ch(\eta) = \sum_{i = 1}^d \chi_{\eta}(e_i) w(e_i).
\]

\textbf{Теорема 2.}

Пусть  $\eta = (e_1, ..., e_d)$ ~--- это некоторый путь на метаграфе $G$, его первая вершина ~--- $u$, а последняя ~--- $v$. И пусть на расширенном графе ему соответствует путь $\mu = (e'_1, ..., e'_d)$.

Если первая вершина в пути $\mu$ ~--- $u^{(j)}$, то последняя вершина этого пути ~--- $v^{(j + ch(\eta)\mod{r})}$.

\textbf{Доказательство.}

Доказательство проведём по индукции по длине пути $\eta$.

Пусть $|\eta| = 1$ и  $\eta = (e) = ((a,b))$, и $ch(\eta) = w(e)$. Дуге $e$ на метаграфе соответствуют дуги $R^{(r)}_e = \{ (a^{(j)}, b^{(j + w(e) (\mod{r}))} ) | j = 1, ..., r \}$
на расширенном графе. Таким образом, конечная вершина каждого пути $\mu_j$ ~--- $b^{(j + ch(\eta) (\mod{r}))}$.

Пусть предположение теоремы верно для всех путей короче $d$.

Пусть $|\eta| = d$ и $\eta = (e_1, ..., e_d)$.

Рассмотрим путь $\epsilon = (e_1, ..., e_{d-1})$. Пусть его последняя вершина ~--- $c$. Тогда по предположению индукции последние вершины соответствующих ему путей равны $c^{(j + ch(\epsilon) (\mod{r}))}$.

Тогда последние вершины путей, соответствующих $\eta$ будут равны $c^{(j + ch(\epsilon) + (-1)^d w(e_d) (\mod{r}))} = c^{(j + ch(\eta) (\mod{r}))}$.

\qed

Очень важными следствиями из Теоремы 2 являются две следующих теоремы.

\textbf{Теорема 3.} \textsl{(О циклах)}

Пусть $\eta$ ~--- цикл на метаграфе. Тогда соответствующие ему пути на расширенном графе являются циклами тогда и только тогда, когда $ch(\eta) = 0$.

\textbf{Теорема 4.} \textsl{(О <<крыльях>> цикла)}

Пусть $\eta = (e_1, ..., e_d)$ и $\mu = (\epsilon_1, ..., \epsilon_{\delta})$ ~--- различные пути на метаграфе с начальной вершиной $a$, конечной вершиной $b$ и $ch(\eta) = \ch(\mu) (\mod{r})$. Тогда путь на расширении, соответствующий склейке этих путей $\gamma = (e_1, \dots, e_d, \epsilon_{\delta}, \dots, \epsilon_1)$, является циклом.

Эти теоремы дают метод нахождения циклов на расширенном графе, описываемый следующим алгоритмом:

\textbf{Алгоритм 1.}

Поиска цикла с длинной меньше или раной заданному числу $l \in \mathbb{N}$.

Пусть $G = \langle A \cup B, E,w\rangle$ ~--- метаграф. Для каждой вершины $a \in A:$

\begin{itemize}
    \item Пометим $a$ парой $(0, 0)$.
    \item В цикле по поколениям $g$ от $1$ до $l$:
      \begin{itemize}
      \item Для всех меток $label = (ch, g - 1)$:
        \begin{itemize}
        \item Вершину, которая помечена меткой $label$, обозначим $v$.
        \item Для всех дуг $e: v \in f(e)$:
          \begin{itemize}
          \item $ch' = ch + (-1)^{g} w(e)$
          \item Обозначим $v'$ вершину, такую, что $v' \in f(e), v' \neq v$.
          \item Если $v'$ ещё не была помечена меткой $(ch', g') \forall g'$ ~--- пометим её парой $(ch', g)$.
          \item Если $v'$ помечена помечена парой $(ch', g)$ ~--- сообщаем о том, что найден цикл длины $g * 2$ и завершаем работу алгоритма.
          \end{itemize}
        \end{itemize}
      \end{itemize}
    \item Сообщаем о том, что цикл не найден и завершаем работу алгоритма.
\end{itemize}

$/*$ Надо добавить пример, сейчас он есть только на бумажечке $*/$

\pagebreak
\section*{Построение графов с заданным обхватом}

$/*$ Слова о задаче. $*/$

% Метод неопределённых коэффициентов

Пусть $G = <V, E, w>$ ~--- метаграф, причём $V = A \cup B$ и $w(e_i) = x_i \forall i = 0..|E|$.

% ----------- Составим систему неравенств, содержащую все неравенства вида $ch(\eta) \neq ch(\mu) : a \in A, v \in V, \eta, \mu$ ~--- несовпадающие пути из $a$ в $v$ равной длины $\leq l$.

Для каждой пары вершин $а$ и $v$, таких что  $а \in A$ и $v \in V$, рассмотрим все неравенства вида $ch(\eta) \neq ch(\mu)$ Где $\eta$ и $\mu$ ~--- различные пути из $a$ в $v$, такие что $|a| = |v| \leq l$.

Систему, содержащую все такие неравенств представим в виде (1).

\begin{equation}
    \centering
    \left\{
        \begin{array}{ll}
            a_{1,1} x_1 + ... + a_{1,e} x_e + b &\neq 0\\
            ...\\
            a_{l,1} x_1 + ... + a_{l,1} x_e + b &\neq 0\\
        \end{array}
    \right.
    \label{eqs:example}
\end{equation}

\textbf{Теорема 5.}

% ------------ Переписать в виде критерия

У системы неравенств (1) есть решение, тогда и только тогда, когда существует расширение метаграфа с обхватом не меньше $2l$.

\textbf{Доказательство.}

Из теоремы 4 следует, что характеристики путей, начинающихся и заканчивающихся в одной вершине равны, т. и т.т., когда они образуют цикл.

% ------------ При построении системы (1) мы рассматриваем пути из всех вершин множетсвао А, всякой длины меньше или равной l. ледовательно, мы рассматриваем все циклы, проходящие через хотя бы одну вершину из множества А и длиной меньше или равной 2l. И других циклов с такой длиной на метаграфе нет.

При построении системы (1), ичерпываются все пути, которые могли бы образовать цикл. Таким образом если у неё есть решение ~--- у метаграфа есть расширение с требуемым обхватом.

\qed

% Ещё слов о задаче

% ------------ Пару фраз о том, что система вида 1 модет быть построена с помощью вот такого алгоритма

Для построения системы вида (\ref{eqs:example}) для заданного метаграфа $G$ можно воспользоваться следующим алгоритмом, являющимся модификацией Алгоритма 1.

\textbf{Алгоритм 2.}

Пусть $G = \langle A \cup B, E,w\rangle$ ~--- метаграф причём $w(e_i) = x_i \forall i = 0..|E|$. Для каждой вершины $a \in A:$

\begin{itemize}
    \item Пометим $a$ парой $(0, 0)$.
    \item В цикле по поколениям $g$ от $1$ до $l$:
      \begin{itemize}
      \item Для всех меток $label = (ch, g - 1)$:
        \begin{itemize}
        \item Вершину, которая помечена меткой $label$, обозначим $v$.
        \item Для всех дуг $e: v \in f(e)$:
          \begin{itemize}
          \item $ch' = ch + (-1)^{g} w(e)$
          \item Обозначим $v'$ вершину, такую, что $v' \in f(e), v' \neq v$.
          \item Если $v'$ ещё не была помечена меткой $(ch', g') \forall g'$ ~--- пометим её парой $(ch', g)$.
          \end{itemize}
        \end{itemize}
      \end{itemize}
    \item Возвращаем систему, содержащую все неравенства вида, $ch_1 \neq \ch_2$ для всех пар меток на одной и той же вершине с одним и тем же поколением.
\end{itemize}

$/*$ Пример $*/$

\section*{Решение систем неравенств}

% ------------ Сказать о том, что это за система вообще (сослаться на прошлую главу)

% ------------ Данный алгоритм можно использовать для поиска решения системы неравенств вида (\ref{eqs:example}).

Поиск минимального решения для систем неравенств вида (\ref{eqs:example}), представленных в прошлой главе, ~--- вычислительно трудная задача. Она предполагает перебор всех возможных решений с выбором минимального. Однако неоптимальное решение можно найти гораздо быстрее, воспользовавшись следующим алгоритмом.

% \textbf{Теорема 6.}

% Решение системы (1) может быть найдено с помощтью алгоритма 3.

\textbf{Алгоритм 3.}

\begin{itemize}
    \item Если в системе содержится неравенство вида $0 \neq 0$, то система не имеет решений, алгоритм завершается с ошибкой.
    \item Найдём все неравенства  вида $a x_e + b = 0$. Обозначим множество таких неравенств $I$.
    \item Выберем число $v: v \neq -b/a \forall a x_e + b \in I$. Такое число точно можно найти, так как неравенств конечное количество. В результате этой замены не возникнет неравенства вида $0 \neq 0$ по условию выбора $v$.
    \item Заменим во всех неравенствах $x_e$ на $v$ и выведем строку $x_e = v$.
    \item Если $e = 1$ ~--- завершим работу алгоритма.
    \item Запустим алгоритм для полученной системы неравнеств с $e - 1$ переменной. Таким образом алгоритм рекурсивно зафиксирует значения всех неизвестных, найдя решение.
\end{itemize}

\textbf{Доказательство.}

% Можно добавить в сам алгоритм слова о том, почему выбор v корректен и доказательство тогда написать "по построению"

% Доказательство проведём по индукции по количеству переменных $e = 1, ...$.

% \begin{itemize}
% \item При $e = 1$ все неравенства имеют вид $a_j x_1 \neq b_j$. Результатом работы алгоритма будет число $v: v \neq -b_{j}/a_{j, 1} \forall j$. 
% \item Предположим, что алгоритм корректно работает для систем с $e <= k - 1$ переменными.
% \item Алгоритм заменит $x_e$ на $v: v \neq -b_{j}/a_{j, e}$ для всех неравенств вида $a_{j, e} x_e + b_j \neq 0$, такое число точно можно найти, так как неравенств конечное количество. В результате этой замены не возникнет неравенства вида $0 \neq 0$ по условию выбора $v$. Затем запустится алгоритм поиска решения неравенства с $e - 1$ переменными, который работает корректно по предположению индукции.
% \end{itemize}

$/*$ Пример $*/$

\qed

% Глава про количество вершин
% Замечание о том, что алгоритм не обязательно находит лучшее возможное решение
% Попробовать доказать, что поиск кратчайшего решения -- НП-полнаязадача

\end{document}